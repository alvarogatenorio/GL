\chapter{Mierdas varias}

En este apéndice irán, de momento, los conceptos sin fundamentar ( y mal explicados).

Comencemos con un ejemplo de ecuaciones paramétricas, completado con cierta interpretación del espacio proyectivo no formalizada aún.

\begin{exa}[Parametrización de una Recta Concreta]
	Dados los puntos $P=(1:2:-1)$ y $Q=(0:1:3)$ se nos pide parametrizar la recta $PQ$. Siguiendo los pasos expuestos en este apartado, la ecuación paramétrica de la recta $PQ$ queda:
	\[
	PQ:\{(\theta:2\theta+1:-\theta+3)\tq\theta\in\proy^1\}
	\]
	donde, cuando $\theta=\infty$, nos referimos al punto $P=(1:2:-1)$.
	
	Imaginemos que ahora queremos hacernos una idea de donde se encuentra esa recta en $\R^3$, es decir, queremos ``pintar'' los rayos de esa variedad proyectiva de dimensión uno. Para ello, debemos escoger un representante afín, y los rayos serán las rectas que vayan desde el $(0,0,0)$ hasta el punto de corte de los vectores representantes de la recta proyectiva con ese plano. Así, además, determinamos donde se encuentran los puntos del infinito del espacio proyectivo. Si elegimos el plano $z=1$, entonces los puntos del infinito estarán en el plano $xy$. 
	
	Elegimos pues el plano $z=1$ como representante afín. Para poder representar los rayos de nuestra recta proyectiva debemos determinar su punto de corte con el plano $z=1$. Por ello dividimos entre $z$. Obtenemos así las ecuaciones
	\begin{equation*}
		x=\frac{\theta}{-\theta+3}, \quad y=\frac{2\theta+1}{-\theta+3}, \quad z=1
	\end{equation*}
	Nótese que hay dos indeterminaciones. Cuando $\theta=\infty$, como ya dijimos, nos referimos al punto $P$, que al dividir entre $z$ nos da el vector representante $(-1,-2,1)$. Cuando $\theta=3$, entonces $z=0$ y nos vamos al plano $xy$, al infinito.
\end{exa}

Pasamos al apartado de ecuaciones implícitas, donde esta misma idea se utiliza para expresar una recta, y en la siguiente observación un plano, a partir de su ecuación implícita en coordenadas no homogéneas.

\begin{obs}
	\label{A2_obs_eq_implicita_nohom}
	Recordemos que, al describir una recta por sus ecuaciones paramétricas, lo hicimos a través de coordenadas homogéneas y no homogéneas. Se puede hacer lo mismo con la ecuación implícita. Observemos que cualquier punto de coordenadas $(x,y)$ queda definido, salvo una constante de proporcionalidad, por la terna $(z_0x,z_0y,z_0)$, con $z_0\not=0$. Asimismo, cualquier terna $(z_0x,z_0y,z_0)$, con $z_0\not=0$, o sus proporcionales, determina un único punto de coordenadas $(x,y)=(\frac{z_0x}{z_0},\frac{z_0y}{z_0})$.
	\begin{equation*}
		\begin{array}{ccccc}
			\R^2&\hookrightarrow&\R^3&\rightarrow &\proy(\R^3)=\proy^2\\
			(x,y)&\rightarrow &(z_0x,z_0y,z_0)&\rightarrow &(z_0x:z_0y:z_0)\\
			(\frac{x}{z},\frac{y}{z})&&\longleftarrow &&(x:y:z)
		\end{array}
	\end{equation*}
	Por tanto, las coordenadas homogéneas $(x,y,z)$ de la ecuación implícita pasan a ser las coordenadas no homogéneas $(X,Y)=(\frac{x}{z},\frac{y}{z})$, quedando así la ecuación de la recta
	\begin{equation}
		\label{A2_eq_implicita_nohom}
		aX+bY+c=0
	\end{equation}
	donde no están incluidos los puntos con $z=0$.\\
\end{obs}

\begin{obs}
	Recordemos que podíamos describir la recta a través de la ecuación implícita en coordenadas no homogéneas. En este caso, esto también es posible. Generalizando la observación~\ref{A2_obs_eq_implicita_nohom} a nuestro caso, es decir a $\proy^3=\proy(\R^4)$, la ecuación implícita del plano en coordenadas no homogéneas quedaría
	\begin{equation}
		\label{A2_eq_implicita_plano_nohom}
		aX+bY+cZ+d=0
	\end{equation}
	donde 
	\begin{equation*}
		X=\frac{x}{t};\qquad Y=\frac{y}{t};\qquad Z=\frac{z}{t}
	\end{equation*}\\
\end{obs}