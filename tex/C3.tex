\chapter{Ecuaciones de rectas y planos proyectivos (por ejemplo)}
\label{C3}

Nuestra tarea aquí es tratar de, dado un subespacio proyectivo, por ejemplo una recta o un plano, dar una referencia proyectiva de ese subespacio mediante la cual hacer una descripción explícita de sus elementos. Comenzaremos estudiando el caso más sencillo, las rectas proyectivas.

\section{Rectas Proyectivas}
\begin{defi}[Recta en $\proy(E)$]
	\label{C1_def_rectaProyectiva}
	Se define \ti{recta proyectiva} que pasa por los puntos proyetivos $P$ y $Q$ como la variedad engendrada por dichos puntos. A dicha recta se la denomina \ti{recta} $PQ$.
\end{defi}
\subsection{Ecuación paramétrica}

Sean $P=\class{u}$ y $Q=\class{v}$ dos puntos proyectivos, vamos a describir los elementos de la recta $PQ$, que no es otra cosa que $\engen{P,Q}$.
	
Para describir los elementos de esta variedad (o de cualquiera) deberemos dar una referencia en función de la cual \ti{coordenar} todos los puntos de la misma.
	
Como $P$ y $Q$ son dos puntos proyectivos distintos, los vectores $u,v$ son linealmente independientes, formando una base de la variedad lineal $\lengen{u,v}$.
	
Para construir una referencia bastaría tomar los puntos $P,Q$ y añadirle como punto unidad un tercer punto cuyo representante pueda ser escrito como combinación lineal de $u$ y $v$ con todos los coeficientes no nulos, por ejemplo $\class{u+v}$.
	
De esta forma tenemos la referencia:
\[\mf{R}=\{P,Q;\class{u+v}\}\]
Por el método de construcción de bases asociadas tenemos que la base asociada a esta referencia es $\mc{B}=\{u,v\}$. Como sabemos, todo punto $p\in\engen{P,Q}$ es un rayo representado por un vector de $\lengen{u,v}$. Es decir, un vector $w=\alpha u+\beta v$ con alguno de los coeficientes no nulo.
	
Esto quiere decir que todo punto de la recta $PQ$ es un rayo de la forma: \[\class{\alpha u+\beta v}=(\alpha:\beta)\]
Sin embargo, podemos reducir esto aún un poco más, cambiemos el representante del rayo dividiendo todo por $\beta$.
\[\class{\frac{\alpha}{\beta}u+v}:\stackrel{\textrm{not.}}{=}\class{\theta u+v}\]
De esta forma la recta ya no queda descrita por dos coordenadas homogéneas $\alpha$ y $\beta$ como antes, sino por una única coordenada $\theta$ a la que llamaremos \ti{no homogénea}.
	
Sin embargo, hemos de tener cuidado, pues, como más de uno ya se habrá dado cuenta, es posible que en algunos casos $\beta$ se anule, por ende, $\theta$ no estaría definida. Como este caso se corresponde con un único punto, y este es el punto $P$, diremos que una recta queda descrita por lo siguiente:
\begin{equation}
	\label{C3_eq_descripcionRecta}
	PQ:\{\class{\theta u+v}\tq \theta\in\K\}\cup\{P\}
\end{equation}
De esta forma, cuando $\beta=0$ podemos decir que $\theta=\infty$, y así $\theta\in\K\cup \{\infty\}$, que podemos identificar con $\proy^1$. Describiremos pues la recta como
\begin{equation}
	\label{C3_eq_descripcionRectaP1}
	PQ:\{\class{\theta u+v}\tq \theta\in\proy^1\}
\end{equation}
donde se entiende que si $\theta=\infty$ nos estamos refiriendo al punto $P$. 

Dados los vectores $u=(u_0,u_1,\cdots,u_n)$ y $v=(v_0,v_1,\cdots,v_n)$ si los sustituimos en la ecuación~\eqref{C3_eq_descripcionRectaP1} obtenemos
\begin{equation*}
	\label{C_3_eq_parametrica_recta}
	\begin{split}
		PQ:&\{\class{\theta (u_0,u_1,\cdots,u_n)+(v_0,v_1,\cdots,v_n)}\tq \theta\in\proy^1\}=\\
		&=\{\class{(\theta u_0+v_0,\theta u_1+v_1,\cdots,\theta u_n+v_n)}\tq \theta\in\proy^1\}
	\end{split}
\end{equation*}
que se puede escribir a su vez como 
\begin{equation}
	PQ:\{(\theta u_0+v_0:\theta u_1+v_1:\cdots:\theta u_n+v_n)\tq \theta\in\proy^1\}
\end{equation}
denominada \tb{ecuación paramétrica de la recta}. Así, la recta proyectiva está formada por todos aquellos puntos proyectivos que cumplan dicha ecuación, es decir, los que se obtienen al ir variando el valor de $\theta$.
\begin{exa}[Parametrización de una Recta Concreta]
	\label{C3_exa_rectaConcreta}
	Dados los puntos $P=(1:2:-1)$ y $Q=(0:1:3)$ se nos pide parametrizar la recta $PQ$. Siguiendo los pasos expuestos en este apartado, la ecuación paramétrica de la recta $PQ$ queda:
	\[
	PQ:\{(\theta:2\theta+1:-\theta+3)\tq\theta\in\proy^1\}
	\]
	donde, cuando $\theta=\infty$, nos referimos al punto $P=(1:2:-1)$.
	
	Imaginemos que ahora queremos hacernos una idea de donde se encuentra esa recta en $\R^3$, es decir, queremos ``pintar'' los rayos de esa variedad proyectiva de dimensión uno. Para ello, debemos fijar un plano y los rayos serán las rectas que vayan desde el $(0,0,0)$ hasta los vectores representantes de la recta proyectiva en ese plano. Así, además, determinamos donde se encuentran los puntos del infinito del espacio proyectivo. Si elegimos el plano $z=1$, entonces los puntos del infinito estarán en el plano $xy$. Para poder representar los rayos de nuestra recta proyectiva debemos determinar su punto de corte con el plano $z=1$. Por ello dividimos entre $z$. Obtenemos así las coordenadas
	\begin{equation*}
		x=\frac{\theta}{-\theta+3}, \quad y=\frac{2\theta+1}{-\theta+3}, \quad z=1
	\end{equation*}
	siendo los puntos de la recta proyectiva los rayos engendrados por los vectores con esas coordenadas. Nótese que hay dos indeterminaciones. Cuando $\theta=\infty$, como ya dijimos, nos referimos al punto $P$, que al dividir entre $z$ nos da el vector representante $(-1,-2,1)$. Cuando $\theta=3$, entonces $z=0$ y nos vamos al plano $xy$, al infinito.
\end{exa}

\subsection{Ecuación implícita}
Durante este apartado nos situaremos en el plano proyectivo $\proy^2$, donde las rectas son hiperplanos. Recordemos que toda recta, variedad proyectiva de dimensión uno, es proyección de un subespacio vectorial de dimensión 2, un plano. Dado que en este caso ese plano pertenece a $\R^3$, será un hiperplano. En realidad esto ya era sabido, todo hiperplano proyectivo es proyección de un hiperplano vectorial. Pero, además, que sea un hiperplano vectorial implica que tiene una, y solo una, ecuación implícita. Y, para no complicarnos la vida, esta será la ecuación de mi hiperplano proyectivo, en este caso una recta. 
Conclusión, al ecuación implícita de una recta en $\proy^2$ tiene la forma
\begin{equation*}
	ax+by+cz=0
\end{equation*}
con $a,b,c$ no todos nulos. 

Veamos pues, varias formas de obtener la ecuación implícita de una recta proyectiva que pasa por dos puntos, es decir, de hallar esos coeficientes.

Sean entonces $P=[u]=[(u_1,u_2,u_3)]$ y $Q=[v]=[(v_1,v_2,v_3)]$ dos puntos proyectivos, y sea la ecuación implícita de la recta $PQ$
\begin{equation*}
	ax+by+cz=0
\end{equation*}
donde $a,b,c$ son coeficientes a determinar. Ya vimos que el conjunto
\[\mf{R}=\{P,Q;\class{u+v}\}\]
es una referencia proyectiva de la recta $PQ$, cuya base asociada es $\mc{B}=\{u,v\}$.

Dado que $P,Q\in PQ$, la primera forma de hallar esos coeficientes consiste simplemente en sustituir en $x,y,z$ de la ecuación implícita las coordenadas de un vector representante de $P$ y de uno de $Q$
\begin{equation*}
	\begin{split}
		au_1+bu_2+cu_3=0\\
		av_1+bv_2+cv_3=0
	\end{split}
\end{equation*}
y resolver el sistema de ecuaciones resultante.

Sin embargo, este método puede resultar un poco tedioso. Observemos que, aunque en el espacio proyectivo no está definido el producto escalar, la primera ecuación del sistema podría identificarse con el producto escalar entre el vector $(a,b,c)$ y $(u_1,u_2,u_3)$. Al ser cero, esto implicaría que son perpendiculares. A partir de la segunda ecuación podemos deducir algo similar, que $(a,b,c)$ es perpendicular al vector $(v_1,v_2,v_3)$.

De esta forma, el vector que debemos hallar es perpendicular a $u$ y $v$. Por tanto nos basta con hallar un vector perpendicular a ambos para determinar los coeficientes de la ecuación implícita, pues es única salvo múltiplos?. Una forma rápida de hallar un vector $(a,b,c)$ que cumpla esto es hacer el producto vectorial de $u$ y $v$. Así los coeficientes serán el resultado de 
\begin{equation*}
	(a,b,c)=u\times v=\left| \begin{array}{ccc}
		x & y & z\\
		u_1 & u_2 & u_3\\
		v_1 & v_2 & v_3
	\end{array}\right| 
\end{equation*}
por lo que la ecuación implícita de la recta vendrá dada por 
\begin{equation}
	(u\times v)\left( \begin{array}{c}
	x\\
	y\\
	z
	\end{array}\right) =0
\end{equation}
