\chapter{Dualidad}
\label{C2}
El objetivo de capítulo es enunciar y justificar el llamado \ti{principio de dualidad}, tanto para espacios vectoriales como para espacios proyectivos.

No debe alarmarse el lector al que no le quedaron claros los conceptos referentes al espacio dual en un primer curso de álgebra lineal, ya que la mayoría se vulve a explicar aquí con todo detalle.

Además, hemos evitado de uno de los conceptos que suele quedar menos claro, las ``bases duales''. No obstante, este es necesario para la demostración de algunos resultados complementarios incluídos en el apéndice \ref{A}, donde se revisitan las bases duales.
\section{Dualidad en espacios vectoriales}
\label{C2_dualidadVectoriales}
Sea $E$ un espacio vectorial de dimensión finita. Llamábamos \ti{espacio dual} asociado a $E$ al conjunto de todas las aplicaciones lineales (homomorfismos de espacios vectoriales) que nacen en $E$ y mueren en $\K$. A estas aplicaciones lineales las denominamos \ti{formas lineales} o $1$--\ti{formas}.

Al espacio dual asociado a $E$ lo denotábamos por $E^*$. Expresado con una notación conjuntista:
\begin{equation}
	\label{C2_eq_defDual}
	E^*:\equals{def.}\{\alpha:E\rightarrow \K\tq \alpha \text{ es lineal }\}=\mathrm{Hom}(E,\K)
\end{equation}
Recordamos sin detenernos que el la palabra \ti{espacio} no le queda grande al conjunto $E^*$ ya que es un espacio vectorial, siendo su dimensión la dimensión de $E$.
\subsection{Formas Lineales e Hiperplanos}
En esta sección iniciaremos la construcción del puente entre un espacio vectorial y su dual, tratando de ``unir'' las variedades más notables de un espacio vectorial, los hiperplanos, con otras variedades de su dual.
 
Antes de comenzar, fijemos una base $\mc{B}$ de $E$. Asimismo fijamos la base $\{1\}$ de $\K$.

Es claro que una forma lineal $h\in E^*$, tiene por matriz asociada cierta matriz $1\times n$ (respecto de las base $\mc{B}$ y $\{1\}$), a esta matriz la  denotaremos símplemente $M$.

Como ya sabemos, para cada vector $u\in E$, el valor $h(u)$ viene dado por:
\begin{equation}
	\label{C2_eq_expMatricialFormaLineal}
	h(u)=MX	
\end{equation}
Siendo $X$ la matriz columna compuesta por las coordenadas de $u$ en la base $\mc{B}$. Es decir, la expresión anterior \eqref{C2_eq_expMatricialFormaLineal} no es más que el producto de una matriz fila por una matriz columna, desarrollémoslo:
\begin{equation}
	\label{C2_eq_expMatricialFormaLinealExpandida}
	h(u)=\begin{pmatrix}
	h(e_1) & \cdots & h(e_n)
	\end{pmatrix}\begin{pmatrix}
	u_1\\
	\vdots\\
	u_n
	\end{pmatrix}=h(e_1)u_1+\dots+h(e_n)u_n
\end{equation}
Como sabemos (es una comprobación inmediata), el kernel, o núcleo, de una aplicación lineal cualquiera, es un subespacio vectorial del espacio de donde nace.

Refrescamos asimismo la fórmula de las dimensiones para aplicaciones lineales, que nos será de utilidad en el futuro inmediato:
\begin{equation}
	\label{C2_eq_grassmannDimensiones}
	\dim(\ker(h))+\dim(\mathrm{im}(h))=\dim(E)
\end{equation}
Nótese que en el contexto de las formas lineales, gracias a la fórmula de las dimensiones \eqref{C2_eq_grassmannDimensiones}, los posibles valores de $\dim(\mathrm{im}(h))$ (a veces denominado \ti{rango} de $h$) son $0$ y $1$. 

En el primer caso, estaríamos ante la aplicación lineal idénticamente nula. En caso contrario, la dimensión del kernel de $h$ (a veces denominada \ti{nulidad} de $h$) sería $n-1$. Es decir, el núcleo de la forma lineal $h$ es un hiperplano de $E$.

Esto significa que $\ker(h)$ podrá ser representado mediante una única \ti{ecuación cartesiana}, ya que $\mathrm{codim}(\ker(h))=1$.

La ecuación cartesiana del núcleo de una forma lineal salta a la vista, ya que se extrae de su propia definición. No es otra que:
\begin{equation}
	\label{C2_eq_ecuacionCartesianaKer}
	h(e_1)u_1+\dots+h(e_n)u_n=0
\end{equation}

Esto es evidente ya que esa es la definición del núcleo de $h$. El conjunto de aquellos vectores que, pasados por $h$, se anulan.

Acabamos de probar que el núcleo de toda forma lineal no nula es un hiperplano de $E$. Es decir, hemos construido un puente entre un espacio vectorial y su dual, identificando cada elemento del dual con un hiperplano de espacio original.

Sin embargo es un puente un tanto quebradizo, ya que solo podemos cruzarlo en una dirección (del dual al original), y, además, es posible que varios elementos del dual queden asociados al mismo hiperplano.

Para afianzar mejor este puente, veamos en primer lugar que podemos cruzarlo en el otro sentido, es decir, que todo hiperplano está asociado ``vía núcleos'' a un elemento del dual.

En términos de aplicaciones:
\begin{lem}[Pseudolema de la Correspondencia]
	\label{C2_lem_pseudocorrespondencia}
	La aplicación:
	\[\begin{array}{c}
	E^*\to\mc{H}\\
	h\mapsto \ker(h)
	\end{array}\]
	es sobreyectiva.
	
	$\mc{H}$ denota el conjunto de los hiperplanos de $E$.
\end{lem}
\begin{proof}
	Dado un hiperplano $H$, obtenemos una ecuación cartesiana que lo describa:
	\[\alpha_1x_1+\dots+\alpha_nx_n=0\]
	Donde los $\alpha_i$ representan escalares y los $x_i$ componentes de un vector respecto de la base $\mc{B}$.
		
	Sabiendo por \eqref{C2_eq_ecuacionCartesianaKer} que la ecuación cartesiana del núcleo de una forma lineal es de la forma:
	\[h(e_1)x_1+\dots+h(e_n)x_n=0\]
	Podemos construir la forma lineal asociada al hiperplano $H$ de forma explícita. Basta definir $h$ como la forma lineal que manda el $i$--ésimo vector de la base $\mc{B}$ al escalar $\alpha_i$. Es decir, la forma lineal con matriz asociada:
	\[\begin{pmatrix}
	\alpha_1 & \cdots & \alpha_n
	\end{pmatrix}\]
\end{proof}

Con el lema \ref{C2_lem_pseudocorrespondencia} hemos avanzado un paso más en la construcción del puente entre un espacio y su dual, pudiendo ir del uno al otro y del otro al uno. Sin embargo, todavía queda un problema pendiente. El elemento del dual asociado a un hiperplano no es único (ni mucho menos).

Dado un hiperplano $H$ existen infinitud de formas lineales cuyo núcleo es $H$. Esto es debido a que toda ecuación lineal equivalente a la ecuación cartesiana del hiperplano es también una ecuación cartesiana del hiperplano, a partir de la cual se construye (lema \ref{C2_lem_pseudocorrespondencia}) una forma lineal asociada a $H$ distinta de la primera.

Por ende, ahora nos interesa encontrar relaciones entre las formas lineales con idéntico núcleo, para así agruparlas y conseguir una biyección.

\begin{lem}[Lema de la Correspondencia]
	\label{C2_lem_correspondencia}
	Todas las formas lineales asociadas a un mismo hiperplano $H$ son múltiplos entre si. En términos de aplicaciones:
	\[\begin{array}{c}
	\proy(E^*)\to\mc{H}\\
	\class{h}\mapsto \ker(h)
	\end{array}
	\]
	es biyectiva. Dicho de otra forma, los hiperplanos de un espacio vectorial se indentifican con las rectas de su dual.
\end{lem}
\begin{proof}
	Sean dos ecuaciones cartesianas de $H$:
	\begin{gather}
		H\equiv \alpha_1x_1+\dots+\alpha_nx_n=0\\
		H\equiv \beta_1x_1+\dots+\beta_nx_n=0
	\end{gather}
	Estas ecuaciones serán equivalentes, es decir, tendrán el mismo conjunto de soluciones (el hiperplano $H$). Por ende (proposición \ref{A1_prop_criterioEquivalencia}) serán múltiplos entre sí, es decir:
	\begin{equation}
		\beta_i = \lambda\alpha_i\ \forall i\in\{1,\dots,n\}
	\end{equation}
	Aplicando el método de construcción (lema \ref{C2_lem_pseudocorrespondencia}) de formas lineales asociadas a $H$ respecto de ambas ecuaciones cartesianas se obtienen:
	\begin{gather}
		f\equiv\begin{pmatrix}
			\alpha_1 & \cdots & \alpha_n
		\end{pmatrix}\\
		g\equiv\lambda\begin{pmatrix}
			\alpha_1 & \cdots & \alpha_n
		\end{pmatrix}
	\end{gather}
	Es decir $f=\lambda g$, o lo que es lo mismo $g\in\class{f}$. Esto prueba la inyectividad que queríamos, ya que la sobreyectividad se probó en el lema \ref{C2_lem_pseudocorrespondencia}.
\end{proof}

Este último resultado pone fin a la construcción del puente entre un espacio y su dual. Recapitulando, un hiperplano queda asociado, ``vía núcleos'' a una única recta su dual. Asimismo, cada recta del dual tiene asociada un hiperplano del espacio ``primal''.
\subsection{Dualidad Canónica}

En esta sección tratamos de generalizar lo dicho en el caso anterior. Es decir, trataremos de identificar variedades lineales arbitrarias con variedades lineales del dual correspondiente.

Antes de comenzar consideramos fijadas las bases $\mc{B}$ de $E$ y $\{1\}$ de $\K$, como en la sección anterior. 

En el caso de los hiperplanos, los identificábamos con el conjunto de las formas lineales cuyo núcleo era dicho hiperplano. Dicho de otra forma, el conjunto de las aplicaciones lineales que anulaban todos los vectores del hiperplano.

Siguiendo esta idea, probaremos a identificar una variedad lineal arbitraria con con el conjunto de aplicaciones que anulan dicha variedad. Dicho conjunto es, si recordamos, nuestro viejo amigo el \ti{anulador} de dicha variedad.

\begin{defi}[Anulador de un subespacio vectorial]
	\label{C2_def_anulador}
	Sea $W\subset E$ subespacio vectorial de $E$. Se define el \ti{anulador} de $W$, al que denotaremos por $W^\perp$, como el conjunto formado por las $1$--formas que anulan todos los vectores de $W$. Es decir:
	\begin{equation}
	W^{\perp}:=\{\alpha\in E^*\tq \alpha(u)=0 \ \forall u\in W\}=\{\alpha\in E^*\tq W\subset ker(\alpha)\}
	\end{equation}
\end{defi}
Inmeditamente se comprueba que el anulador de un subespacio vectorial es un subespacio vectorial del dual asociado. 

\begin{obs}[Generalización de la Noción de Anulador]
	Se puede, generalizar la definición de anulador, no solo para subespacios sino tambien para subconjuntos arbitrarios de un espacio vectorial. La conveniencia de esta generalización viene dada porque desbloquea algunos resultados técnicos de gran utilidad. Como la importancia de esto aquí es muy reducida, tanto la generalización de la definición \ref{C2_def_anulador} como los resultados más elementales se han trasladado al apéndice. En concreto a la definición \ref{A1_def_anulador} y al lema \ref{A1_lem_anuladorBase}.
\end{obs}

Intentemos reeditar el lema de la correspondencia \ref{C2_lem_correspondencia} del apartado anterior, tratando de identificar a cada subespacio de $E$ con su anulador correspondiente.
\begin{lem}[Lema de la Correspondencia]
	\label{C2_lem_correspondenciaAnulador}
	Los subespacios de $E$ están en biyección con los subespacios de $E^*$ de la siguiente manera:
	\[\begin{array}{c}
	\mc{U}\stackrel{\Phi}{\to}\mc{U}^*\\
	U\mapsto U^{\perp}
	\end{array}\]
	Donde $\mc{U}$ y $\mc{U}^*$ denotan el conjunto de los subespacios de $E$ y $E^*$ respectivamente.
	Dicho de otra forma, cada subespacio puede puede identificarse con su anulador.
\end{lem}
\begin{proof}
	Para la sobreyectividad veamos que todo subespacio $W$ de dimensión $r$ de $E^*$ es el anulador de un cierto subespacio $U$.
	
	Para esto, trataremos de probar que el conjunto de vectores de $E$ que son anulados por todas las formas lineales de $W$ (imagen inversa de $\Phi$) es un subespacio vectorial de $E$. A este conjunto le denominaremos \ti{antianulador} o \ti{anulador dual}. Este nombre se debe a que el anulador de dicho conjunto (por definición) es $W$.
	
	Notamos que $W$ admitirá una cierta base compuesta de $r$ formas lineales. Esto tiene importancia, ya que, cada vector que sea anulado por todas las formas lineales de la base, también lo será, por linealidad, por todas las aplicaciones de $W$.
	
	Dicho lo cual, tenemos que:
	\[W=\lengen{f_1,\dots,f_r}\]
	Dichas formas lineales tendrán ciertas matrices asociadas:
	\[f_i\equiv\begin{pmatrix}
	a_1^i & \dots & a_n^i
	\end{pmatrix}\]
	Dado un vector $x=(x_1,\dots,x_n)_{\mc{B}}$, su valor por $f_i$ viene dado por la ecuación  \eqref{C2_eq_expMatricialFormaLinealExpandida}:
	\[f_i(x)=a_1^ix_1+\dots+a_n^ix_n\]
	
	Así, pues el conjunto de los vectores tales que son anulados por las formas lineales de $W$ (antianulador) \tb{es} el conjunto de vectores que cumplen las ecuaciones:
	\begin{equation}
	\label{C2_eq_ecuacionesAtianulador}
	\left\{\begin{array}{ccc}
	a_1^1x_1+\dots+a_n^1x_n&=&0\\
	\ddots& \vdots & \vdots\\
	a_1^rx_1+\dots+a_n^rx_n&=&0
	\end{array}\right.\end{equation}
	Estas ecuaciones pueden interpretarse por las ecuaciones cartesianas de cierto subesacio $U$ de dimensión $n-r$ cuyo anulador es precisamente $W$.
	
	Esto también prueba la inyectividad, ya que hemos visto que la imagen inversa de $\Phi$ es un único subespacio.
\end{proof}
\begin{obs}[Dimensiones de los Dualizados]
	\label{C2_obs_dim_anulador}
	Obsérvese que hemos probado algo bastante importante (además de lo queríamos probar en un principio), y es que, las dimensiones de un subespacio y su anulador suman la dimensión del espacio total, es decir:
	\begin{equation}\label{C2_eq_dim_anulador}
	\dim(U)+\dim(U^{\perp})=\dim(E)
	\end{equation}
	
	Dicho de otra forma, un subespacio $U$ de dimensión $r$ ``dualiza'' en un subespacio $U^\perp$ de dimensión $n-r$ (su codimensión).
\end{obs}

Recapitulando, hemos conseguido construir un puente ``sólido'' y ``de ida y vuelta''entre las variedades de un espacio vectorial y las de su dual ``vía anuladores'' y ``vía antianuladores'' respectivamente.

Para el lector que haya optado por omitir la demostración del lema \ref{C2_lem_correspondenciaAnulador} el término ``antianulador'' resultará extraño. Definámoslo aparte:
\begin{defi}[Antianulador]
	Sea un subespacio $W$ del dual $E^*$, se define su \ti{antianulador} como el conjunto de los vectores que son anulados por todos los elementos de $W$. Expresado de forma conjuntista:
	\[W^\top:=\{u\in E\tq \alpha(u)=0\ \forall\alpha\in W\}\]
\end{defi}

Es claro que esto no es más que la imagen inversa de la aplicación del lema \ref{C2_lem_correspondenciaAnulador}, que asigna a cada subespacio de $E$ su anulador correspondiente. Como ese mismo lema demuestra que tal aplicación es biyectiva es trivial deducir que:
\begin{equation}
	\label{C2_eq_involutividad}
	(W^{\perp})^{\top}=W
\end{equation}

\begin{obs}[Abusos de Notación Habituales]
	Algunos textos (y docentes) denotan al antianulador de la misma forma que lo hacen con el anulador. Además, para dar verosimilitud a este abuso de notación, definen el anulador de un subespacio de $E^*$ como lo que nosotros conocemos como antianulador. Dicho abuso de notación les permite hacer afirmaciones como:
	\[(W^\perp)^\perp=W\]
	Lo cual, a priori, con nuestra notación, es una trola bastante grande.
	
	Sin embargo, trabajando con cuidado, podemos demostrar que, efectivamente, con nuestras notaciones se cumple que el operador anulador es ``esencialmente involutivo''. Es decir:
	\begin{equation}
		(W^\perp)^\perp\cong W
	\end{equation}
	Esto significa que existe un isomorfismo entre el anulador del anulador un subespacio (un subespacio del dual del dual) y el subespacio en sí. No solo eso, se puede demostrar que dicho isomorfismo es especialmente agradable, por lo que se le da el nombre de \ti{canónico}.
	
	Como la prueba de este hecho no pinta mucho aquí, se ha trasladado al apéndice \ref{A}.
\end{obs}
Veamos a continuación una serie de propiedades que nos serán de gran ayuda cuando conozcamos el llamado ``principio de dualidad'', en la sección \ref{C2_principioDualidadLineal}. Estas propiedades pueden resumirse en dos; la inversión de las contenciones y las leyes de DeMorgan.
\begin{prop}[Propiedades del Anulador]
	\label{C2_pro_propiedades_dualidad}
	Se cumplen las siguientes propiedades:
	\begin{enumerate}
		\item Los contenidos se invierten al dualizar. Es decir: \[W\subset U\sii U^{\perp}\subset W^{\perp}\]
		\item Las sumas se convierten en intersecciones al dualizar:
		\[(U+W)^{\perp}=U^{\perp}\cap W^{\perp}\]
		\item Las intersecciones se convierten en sumas:
		\[(U\cap W)^{\perp}=U^{\perp}+ W^{\perp}\]
	\end{enumerate}
\end{prop}
\begin{proof}
	\begin{enumerate}
		\item \begin{itemize}
			\item[$\bra$] Dado un $\alpha\in U^\perp$, veamos que $\alpha\in W^\perp$. En efecto, $\alpha(u)=0$ para todo $u\in U$, pero como $W\subset U$ se tiene que $\alpha(w)=0$ para todo $w\in W$, luego $\alpha\in W^\perp$.
			
			\item[$\bla$] Sea $w\in W$, veamos que $w\in U$. Es claro que $\alpha(w)=0$ para toda $\alpha \in W^{\perp}$. Como $U^{\perp}\subset W^{\perp}$ se tiene que $\beta (w)=0$ para toda $\beta\in U^{\perp}$. Luego $w$ pertenece al antianulador del anulador de $U$. Por la ecuación \eqref{C2_eq_involutividad} se tiene que $w\in (U^\perp)^\top=U$.
		\end{itemize}
		
		
		\item Sea $\alpha\in(U+W)^\perp$, inmediatamente se desprende que $\alpha(u+w)=\alpha(u)+\alpha(w)=0$ para todo $u\in U$ y todo $w\in W$. Como $u\in U+W$ y $w\in U+W$ tenemos que $\alpha(u)=0$ para todo $u\in U$ y $\alpha(w)=0$ para todo $w\in W$. Luego $\alpha$ pertenece a los anuladores de $U$ y $W$ simultáneamente. Es decir, $\alpha\in U^\perp\cap W^\perp$. Como todos los pasos que hemos hecho son equivalencias, el resultado se sigue.
		\item Dado $\alpha\in U^\perp + W^\perp$, veamos que $\alpha\in (U\cap W)^\perp$. Usando que $\alpha\in U^\perp + W^\perp$ tenemos que $\alpha=\beta+\gamma$ donde $\beta\in U^\perp$ y $\gamma\in W^\perp$. Para probar que pertenece al anulador de la intersección, tomemos un $\xi\in U\cap W$ arbitrario y veamos que $\alpha$ lo anula. En efecto:
		\[\alpha(\xi)=(\beta+\gamma)(\xi)=\beta(\xi)+\gamma(\xi)=0\]
		Acabamos de ver que $U^\perp+W^\perp\subset(U\cap W)^\perp$. Para ver la igualdad, veamos que ambos tienen la misma dimensión. Para ello usaremos la fórmula de Grassmann, el apartado anterior de esta demostración y la observación \ref{C2_obs_dim_anulador}:
		\begin{multline}
			\dim(U^\perp+W^\perp)=\dim(U^\perp)+\dim(W^\perp)-\dim(U^\perp\cap W^\perp)=\\
			=\dim(U^\perp)+\dim(W^\perp)-\dim((U+ W)^\perp)=\\
			=n-(\dim(U)+\dim(W)-\dim(U+W))=\\
			=n-(\dim(U\cap W))=\\
			=\dim((U\cap W)^\perp)
		\end{multline}
	\end{enumerate}
	Con lo que concluye la demostración.
\end{proof}

A la identificación de un subespacio con su anulador se la denomina \ti{dualidad canónica}.
\subsection{Principio de Dualidad}
\label{C2_principioDualidadLineal}

Estamos en disposición de enunciar el llamado \ti{principio de dualidad}, un resultado de extrema importancia, ya que, por cada teorema que demostremos obtendremos otro sin necesidad de demostrarlo. A continuación trataremos de enunciar y justificar este principio de forma natural, con un ejemplo. Presentamos los siguientes enunciados:
\begin{theo}
	\label{C2_teo_principioDualidad1}
	Sea un espacio vectorial de dimensión $3$ entonces:
	
	Dos rectas distintas generan un plano.
\end{theo}
\begin{theo}
	\label{C2_teo_principioDualidad2}
	Sea un espacio vectorial de dimensión $3$ entonces:
	
	Dos planos distintos se cortan en una recta.
\end{theo}
A simple vista los teoremas \ref{C2_teo_principioDualidad1} y \ref{C2_teo_principioDualidad2} parecen dos teoremas totalmente independientes de tal forma que cada cual requerirá una prueba.

Bien, tratemos de demostrar el teorema \ref{C2_teo_principioDualidad1}. (Recomendamos encarecidamente al lector no omitir la siguiente demostración).
\begin{proof}[Demostración del Teorema \ref{C2_teo_principioDualidad1}]
	Nos preguntamos si la suma de dos subespacios de dimensión uno cuya intersección es un espacio de dimensión nula tendrá dimensión $2$. Es decir:
	\begin{equation*}
		\dim(U+W)\stackrel{\mathrm{?}}{=}2
	\end{equation*}
	
	Como por la observación \ref{C2_obs_dim_anulador} preguntarse por la dimensión de un subespacio es preguntarse por la dimensión de su anulador obtenemos que nuestra pregunta inicial es equivalente a:
	\begin{equation*}
		\dim((U+W)^\perp)\stackrel{\mathrm{?}}{=}3-2=1
	\end{equation*}
	
	Asimismo, por la proposicion \ref{C2_pro_propiedades_dualidad} sabemos que el anulador de una suma es la intersección de anuladores, luego esta segunda pregunta es equivalente a:
	\begin{equation*}
		\dim((U^\perp\cap W^\perp))\stackrel{\mathrm{?}}{=}1
	\end{equation*}
	Donde, por la observación \ref{C2_obs_dim_anulador} tenemos que $\dim(U^\perp)=\dim(W^\perp)=2$.
	
	Además, como $\dim(U\cap W)=0$, se tiene que $\dim(U^\perp+W^\perp)=3$.
	
	Así, a partir de nuestra primera pregunta, hemos obtenido una equivalente que dice reza:
	
	\ti{¿En un espacio de dimensión $3$, dos planos distintos se cortan en una recta?}
	
	Y esta es, precisamente la pregunta que deberíamos hacernos si estuviéramos tratando de demostrar el teorema \ref{C2_teo_principioDualidad2}.
	
	Con esto hemos demostrado que los teoremas \ref{C2_teo_principioDualidad1} y \ref{C2_teo_principioDualidad2} son equivalentes. Es decir, si uno es cierto, es cierto el otro (y viceversa). Asimismo, si uno es falso, el otro también lo será (y al revés). Por ende, con probar uno de los dos nos valdrá.
	
	En lo que respecta a la prueba del teorema, es un cálculo inmediato con la fórmula de Grassmann y se deja como ejercicio de cálculo mental al lector.
\end{proof}

Quizá este no sea el mejor ejemplo para apreciar la gran utilidad de este principio, ya que, las demostraciones de ambos teoremas son extraordinariamente sencillas. Sin embargo, pueden darse casos (y se derán a lo largo del texto) en los que la demostración de un teorema sea extraordinariamente sencilla en el caso dual y algo más engorrosa en el caso ``normal''.

Así pues, dado cierto aserto sobre espacios vectoriales compuesto en términos de sumas, intersecciones y contenidos puede traducirse a un \ti{aserto dual} equivalente gracias a las propiedades demostradas en la sección anterior.

Esto es una auténtica fábrica de teoremas, ya que, si demostramos la veracidad de un enunciado, obtendremos automáticamente la veracidad de su aserto dual equivalente. 
\section{Dualidad en espacios proyectivos}
Una vez repasados y ampliados los conceptos de espacio dual y anulador, pasemos a introducir el espacio proyectivo dual. Como siempre, iremos trasladando al contexto proyectivo los resultados del mundo lineal.
\begin{defi}[Espacio proyectivo dual]
	Dado un espacio vectorial $E$ y su correspondiente espacio proyectivo $\proy(E)$, se llama \ti{espacio proyectivo dual} de $P$ al espacio proyectivo $\proy(E^*)$ asociado al espacio vectorial dual $E^*$ de $E$. Se denota por $\proy(E^*)$. En el caso de ser $E=\K^{n+1}$, su espacio proyectivo dual se denota por $\proy^*$.
\end{defi}
\begin{obs}
	Dado que el espacio dual $E^*$ tiene la misma dimensión que $E$, si la dimensión es finita, esto implica que la dimensión del espacio proyectivo es la misma que la del espacio proyectivo dual
	\begin{equation*}
	\dim(\proy(E))=\dim (\proy(E^*))
	\end{equation*}
\end{obs}

\subsection{Formas Lineales e Hiperplanos Proyectivos}

Comenzamos este apartado recordando brevemente que los hiperplanos vectoriales son subespacios lineales de codimensión $1$. Asimismo todo hiperplano proyectivo $H=\proy(\hat{H})$ es la proyección de un hiperplano vectorial $\hat{H}$
\begin{equation}
\dim(E)-2=\dim(\proy)-1=\dim(H)=\dim(\hat{H})-1\ra \dim(\hat{H})=\dim(E)-1.
\end{equation}
Recordemos también que todo hiperplano vectorial $\hat{H}$ está en biyección con las formas lineales cuyo núcleo es el propio espacio vectorial $\hat{H}$, las cuales son múltiplos. Es decir, está en biyección con un punto del espacio proyectivo dual, un rayo de formas lineales.

Por tanto podemos intentar establecer también una biyección entre un hiperplano proyectivo $H$ y dicho punto proyectivo $[h]$, ya que $H$ es la proyección de un hiperplano vectorial. Pero la forma lineal $h$ no está bien definida en el espacio proyectivo $\proy(E)$. Ello se debe a que puedo escoger dos vectores  $u, \lambda u\in E$, que pertenecen al mismo punto en el espacio proyectivo, tales que sus imágenes no pertenecen al mismo rayo. Sin embargo, esto no puede ocurrir con aquellos que pertenezcan al núcleo de $h$. Por tanto los ceros de $h$ sí están bien definidos en $\proy(E)$. Dado que estos ceros son precisamente el espacio vectorial $\hat{H}$, no hay ningún problema en considerar el espacio proyectivo.

Surge así el siguiente lema, idéntico al lema~\ref{C2_lem_correspondencia}.
\begin{lem}[Lema de la Correspondencia proyectiva]
	\label{C2_lem_correspondenciaProy}
	La aplicación
	\begin{equation*}
		\begin{split}
			\mc{H}& \rightarrow \proy(E^*)\\
			H &\rightarrow [h]
		\end{split}
	\end{equation*}
	es biyectiva. $\mc{H}$ denota el conjunto de los hiperplanos proyectivos de $\proy(E)$.
\end{lem}
\begin{proof}
	El conjunto de los hiperplanos proyectivos de $\proy(E)$ está en biyección con el conjunto de los hiperplanos vectoriales de $E$. Estos, a su vez, por el Lema de la Correspondencia, están en biyección con el espacio proyectivo dual, asociando a cada hiperplano vectorial $\hat{H}$ el rayo de formas lineales que se anulan sobre él. Dado que la composición de biyecciones es biyección, queda demostrado que existe una aplicación biyectiva que asocia a cada hiperplano proyectivo $H=\proy(\hat{H})$ el rayo de formas lienales cuyos ceros forman el hiperplano  $\hat{H}$.
\end{proof}
Entonces, dado un hiperplano proyectivo $H$, podemos escribirlo como
\begin{equation}
H=\proy(\hat{H})=\{[u]\in\proy(E)\tq h(u)=0\}
\end{equation}

\subsection{Dualidad Canónica}
Se ha visto la importancia del espacio dual a la hora de caracterizar hiperplanos proyectivos. Al igual que se hizo en la sección anterior, buscamos ahora generalizar ese resultado, para poder identificar variedades proyectivas con variedades lineales del espacio dual proyectivo.

Recordemos que, dada una variedad lineal, esta estaba en biyección con su anulador, el cual es un subespacio vectorial de $E^*$. Al igual que hicimos con hiperplanos trataremos de trasladar esta biyección entre variedades de un espacio vectorial y su dual, a una biyección entre variedades de un espacio proyectivo y su dual.
\begin{lem}[Lema de la Correspondencia Proyectiva]
	\label{C2_lem_correspondencia_proy_anulador}
	La aplicación 
	\begin{equation*}
		\begin{split}
			\mc{X}& \rightarrow \mc{X}^*\\
			X& \rightarrow X^*=\proy(\hat{X}^{\perp})
		\end{split}
	\end{equation*}
	donde $\mc{X}$ y $\mc{X}^*$ denotan el conjunto de las variedades de $\proy(E)$ y $\proy(E^*)$ respectivamente, es biyectiva.
\end{lem}
\begin{proof}
	El conjunto de las variedades de $\proy(E)$ está en biyección con los subespacios vectoriales de $E$. Estos, a su vez, por el lema~\ref{C2_lem_correspondenciaAnulador} (Lema de la Correspondencia), están en biyección con su anulador $\hat{X}^\perp$. Al ser este un subespacio vectorial del espacio dual $E^*$, de nuevo podemos establecer una biyección entre $\hat{X}^{\perp}$ y las variedades de $\proy(E^*)$, que asocia a cada subespacio su correspondiente proyección $X^*=\proy (\hat{X}^{\perp})$. Dado que la composición de biyecciones es biyección, queda demostrado que existe una aplicación biyectiva que asocia a cada variedad proyectiva $X$ la proyección de su anulador $X^*$.
\end{proof}

Dada una variedad proyectiva $X$, llamaremos dual de la variedad a $X^*$.

Obsérvese que, lo que estamos haciendo, es identificar cada variedad proyectiva $X=\proy(\hat{X})$ con los rayos de las formas lineales que pertenecen al anulador de $\hat{X}$. Dado que un subespacio vectorial tiene tantas ecuaciones cartesianas como formas lineales independientes hay en su anulador, es fácil hacer esta identificación. Un punto del espacio proyectivo, por ejemplo, se identifica con dos elementos del dual linealmente independientes, es decir, un plano, ya que el subespacio del que es proyección es una recta, la cual posee dos ecuaciones cartesianas.

Por supuesto, el hiperplano es un caso particular de esta caracterización, en la que solo hay una ecuación cartesiana y por tanto la variedad proyectiva se identifica con un único punto en el espacio proyectivo dual. Así
\begin{equation}
H^*=\proy(\hat{H}^{\perp})=\proy(\{h\in E^*\tq h(u)=0\ \forall u\in \hat{H}\})=[h]
\end{equation}

Las propiedades que se desprendían del Lema de la Correspondencia para espacios vectoriales, se dan también entre variedades proyectivas, como muestra la siguiente proposición.
\begin{prop}[Propiedades de la Dualidad Proyectiva]
	\label{C2_pro_propiedades_dualidad_proy}
	Sea $E$ un espacio vectorial y su correspondiente espacio proyectivo $\proy(E)$. Sean $X,Y\subset\proy(E)$ variedades proyectivas. Se cumple
	\begin{enumerate}
		\item Si $X\subset Y$, entonces $Y^*\subset X^*$
		
		\item $(X\cap Y)^*=\engen{X^*,Y^*}$
		
		\item $\engen{X,Y}^*=X^*\cap Y^*$
		
		\item $\dim(X)+\dim(X^*)=\dim(\proy)-1$
	\end{enumerate}
\end{prop}
\begin{proof}
	Sean $X=\proy(\hat{X})$ e $Y=\proy(\hat{Y})$ variedades proyectivas.
	\begin{enumerate}
		\item Si $X\subset Y$, entonces $\hat{X}\subset\hat{Y}$. Por la proposición~\ref{C2_pro_propiedades_dualidad} esto implica que $\hat{Y}^{\perp}\subset \hat{X}^{\perp}$, y por tanto $Y^*\subset X^*$.
		
		\item Por el lema~\ref{C1_lem_interseccionVariedades} se tiene que $X\cap Y=\proy(\hat{X}\cap\hat{Y})$. Por tanto $(X\cap Y)^*=\proy((\hat{X}\cap\hat{Y})^{\perp})$. Aplicando la proposición~\ref{C2_pro_propiedades_dualidad} se tiene que $\proy((\hat{X}\cap\hat{Y})^{\perp})=\proy(\hat{X}^{\perp}+\hat{Y}^{\perp})=\engen{X^*,Y^*}$.
		
		\item Se tiene que $\engen{X,Y}=\hat{X}+\hat{Y}$. Por tanto $\engen{X,Y}^*=\proy((\hat{X}+\hat{Y})^{\perp})$. Por la proposición~\ref{C2_pro_propiedades_dualidad} sabemos que $\proy((\hat{X}+\hat{Y})^{\perp})=\proy(\hat{X}^{\perp}\cap\hat{Y}^{\perp})$. Atendiendo de nuevo al lema~\ref{C1_lem_interseccionVariedades} queda que $\proy(\hat{X}^{\perp}\cap\hat{Y}^{\perp})=\proy(\hat{X}^{\perp})\cap\proy(\hat{Y}^{\perp})=X^*\cap Y^*$
		
		\item Por la ecuación~\eqref{C2_eq_dim_anulador} se tiene que $\dim(\hat{X})+\dim(\hat{X}^{\perp})=\dim(E)=\dim(\proy(E))+1$. Teniendo en cuenta la definición~\ref{C1_def_dimension} queda
		\begin{equation*}
		\dim(X)+\dim(X^*)=\dim(\hat{X})-1+\dim(\hat{X}^{\perp})-1=\dim(\proy(E))+1-2=\dim(\proy(E))-1
		\end{equation*}
	\end{enumerate}
\end{proof}
\begin{obs}
	El lema anterior confirma que el dual de un hiperplano proyectivo es un punto. En efecto supongamos que $\dim(E)=m+1$, entonces
	\begin{equation*}
	\dim(X)+\dim(X^*)=m-1+\dim(X^*)=\dim(\proy(E))-1=m-1\sii \dim(X^*)=0
	\end{equation*}
\end{obs}

\subsection{Principio de Dualidad para espacios proyectivos}
Comencemos con una definición.
\begin{defi}
	Sea $\mc{P}$ una proposición relativa a los subespacios de un espacio proyectivo y formulada en términos de intersecciones, variedades generadas por uniones, contenidos y dimensiones de estos subespacios. Se llama \ti{proposición dual} $\mc{P}'$ a la que se obtiene a partir de $\mc{P}$ sustituyendo los términos anteriores por sus duales.
\end{defi}
Observemos que sustituir un término por su dual no es más que aplicar la proposición~\ref{C2_pro_propiedades_dualidad_proy} en el sentido adecuado. Por ejemplo, hallemos la proposición dual de ``\ti{en un plano proyectivo real toda recta contiene al menos tres puntos diferentes}'' . Dado que un plano proyectivo real $\proy^2$ tiene dimensión $2$, el dual de una recta es un punto y el de un punto una recta
\begin{equation*}
\dim(X)+\dim(X^*)=1. 
\end{equation*}
Por tanto su proposición dual será ``\ti{en un plano proyectivo real por todo punto pasan al menos tres rectas diferentes}''.

Una vez definido este concepto, podemos enunciar un teorema de gran importancia.
\begin{theo}[Principio de Dualidad] Una proposición $\mc{P}$ relativa a variedades proyectivas de espacios de dimensión finita $n$ sobre un cuerpo $\K$ es cierta si y solo si lo es su proposición dual $\mc{P}'$
\end{theo}
\begin{proof}PENDIENTE
	Sea $P$ un espacio proyectivo de dimensión $n$ sobre $\K$ y sea una proposición $\mc{P}$ cierta en $P^*$. Tenemos que demostrar que $\mc{P}'$ es cierta. Esta última se obtiene de sustituir las intersecciones, variedades generadas por uniones, contenidos y dimensiones de estos subespacios que haya en $\mc{P}$ por sus duales. Por lo que si estos eran ciertos en $P^*$, al pasar a su dual, seguirán siendo ciertos en su espacio dual, es decir en $P^{**}=P$, donde se enuncia $\mc{P}'$. Por lo que $\mc{P}'$ es cierta.
	
	El recíproco es análogo.
\end{proof}
La gran importancia de este teorema radica en que permite obtener, sin necesidad de demostración, un nuevo teorema a partir de cada teorema conocido. En efecto dado un teorema, este constituye una proposición $\mc{P}$, y por el principio de dualidad, automáticamente su proposición dual $\mc{P}'$ es cierta, es decir, el dual del teorema, es cierto.

NO SABÍA MUY BIEN DONDE PONER EL PRINCIPIO ESTE, SI DELANTE O DETRÁS DEL EJEMPLO, LO HE PUESTO DELANTE PERO PUEDES CAMBIARLO DONDE TE PLAZCA

Todas estas caracterizaciones no serían de ninguna utilidad si no nos permitiesen resolver problemas de espacio proyectivo con mayor facilidad. Hasta ahora no hemos visto ninguna aplicación. Simplemente hemos ido explicando como se hace ese paso al espacio proyectivo dual, insistiendo una y otra vez en su importancia. Pero ¿realmente es tan importante? ¿No podemos simplemente resolver los problemas en el espacio proyectivo o echando mano del espacio vectorial? Es posible, sí, pero muchas veces hacer la asociación entre una variedad proyectiva y su dual, es decir la proyección del anulador, facilita enormemente la resolución. Veamos a continuación un ejemplo.
\begin{exa}
	Sea $\proy^3=\proy(\R^4)$. Sean dos rectas del espacio proyectivo $r_1,r_2\in\proy^3$, las cuales no se cortan, y un punto $p\in\proy^3$ que no pertenece a ninguna de las rectas. Demuestre que existe una única recta $r\in\proy^3$ que pasa por p y corta a ambas rectas $r_1, r_2$.\\
	
	Según el enunciado del problema tenemos dos rectas $r_1,r_2\in\proy^3$ y un punto $p\in\proy^3$ tales que $r_1\cap r_2=\emptyset$ y $p\not\in r_1\cup r_2$. Debemos probar que existe una única recta $r\in\proy^3$ tal que $p\in r$, $r_1\cap r\not=\emptyset$ y $r_2\cap r\not=\emptyset$. Resolvamos el problema primero sin dualizar, y luego pasando al dual.
	\begin{enumerate}
		\item Tomemos la variedad proyectiva engendrada por $r_1$ y $p$, la cual es un plano ya que
		\begin{equation*}
		\dim(\engen{p,r_1})=\dim(p)+\dim(r_1)-\dim(r_1\cap p)=0+1-(-1)=2
		\end{equation*}
		Podemos aplicar el corolario~\ref{C1_cor_rectaHiperplano} al plano $\engen{p,r_1}$ y la recta $r_2$, según el cual una recta y un hiperplano siempre se cortan. Antes, y para obtener el resultado deseado, debemos asegurarnos de que $r_2\not\subset\engen{p,r_1}$, pues en caso contrario existirían más de un punto de corte entre la recta y el hiperplano y $r$ no sería única. Es fácil comprobar que esto no ocurre, ya que si $r_2\subset\engen{p,r_1}$, entonces $r_1\cap r_2\not=\emptyset$, llegando así a un absurdo. Existirá por tanto un único punto $q\in r_2\cap\engen{p,r_1}$. Definimos entonces la recta $r$ como la variedad engendrada por los puntos $p$ y $q$, pudiéndose comprobar con la fórmula de las dimensiones que efectivamente es una recta. Por un lado $r$ es única, ya que lo es el punto $q$. Además $r_1\cap r\not=\emptyset$ y $r_2\cap r\not=\emptyset$, ya que $q\in r_2\cap\engen{p,r_1}$. Queda así demostrado el ejercicio.
		
		\item Dado que es la primera vez que dualizamos un problema, hagámoslo paso a paso. Para empezar, y atendiendo a la proposición~\ref{C2_pro_propiedades_dualidad_proy}, la ecuación de las dimensiones que caracteriza la dualización es, en nuestro caso,
		\begin{equation*}
		\dim(X)+\dim(X^*)=2. 
		\end{equation*}
		Por tanto el dual de un punto es un plano del espacio proyectivo dual y el dual de una recta, una recta. Tenemos entonces que $p^*$ es un plano y $r_1^*,r_2^*$ son rectas. Por otro lado que $p\in r$ implica, por la proposición~\ref{C2_pro_propiedades_dualidad_proy}, que $r^*\subset p^*$. De igual forma que $p\not\in r_1\cup r_2$ implica que $r_1^*\not\subset p^*$ y $r_2^*\not\subset p^*$. Además si $r_1\cap r_2=\emptyset$, entonces $r_1^*\cap r_2^*=\emptyset$. En caso contrario existiría un plano dual $\pi^*$ tal que $r_1^*\subset\pi^*$ y $r_2^*\subset\pi^*$. Utilizando de nuevo la fórmula de las dimensiones y la proposición~\ref{C2_pro_propiedades_dualidad_proy}, esto equivaldría a decir que existe un punto $q$ tal que $q\in r_1$ y $q\in r_2$, llegando así a un absurdo.
		
		Por tanto el enunciado del problema se traduce en, dadas dos rectas $r_1^*, r_2^*\in\proy^{3^*}$ y un plano $p^*\in\proy^{3^*}$ tales que $r_1^*\cap r_2^*=\emptyset$, $r_1^*\not\subset p^*$ y $r_2^*\not\subset p^*$; demostrar que existe una única recta $r^*$ tal que $r_1^*\cap r^*\not=\emptyset$, $r_2^*\cap r^*\not=\emptyset$ y $r^*\subset p^*$.
		
		Dado que las rectas $r_1^*, r_2$ no están contenidas en el plano $p^*$, cortarán con él en dos puntos únicos. Es claro que la recta engendrada por esos dos puntos es única y cumple las condiciones requeridas.
	\end{enumerate}
\end{exa}
\begin{obs}
	Este enunciado es falso en espacio afín. Podemos encontrar dos rectas paralelas, $r_1$ y $r_2$, y un punto $p$, que cumplan las hipótesis del enunciado, para los cuales no existe ninguna recta $r$; o bien para los cuales existan infinitas rectas $r$, tales que $p\in r$, $r_1\cap r\not=\emptyset$ y $r_2\cap r\not=\emptyset$. Ello se debe a que en el espacio afín dos rectas paralelas no se cortan, mientras que en el espacio proyectivo sí (en el infinito), y por lo tanto no cumplen las hipótesis del enunciado.
\end{obs}
\begin{obs}
	Una vez resuelto este ejercicio podemos observar diferencias en los métodos de resolución. Mientras que en el primer caso hemos tenido que construir la recta sin mucha idea de a donde nos llevaría e ir comprobando que cumple los requisitos, al traducir el problema al espacio dual, la recta ha surgido por sí sola, como consecuencia de las hipótesis del enunciado. Es cierto que, debido a la sencillez de este ejercicio, la diferencia en la dificultad de resolución no es tan clara. Sin embargo, es posible darse cuenta de que, en problemas más complicados, el espacio proyectivo dual nos da un camino más rápido. La única dificultad radica en traducir bien los enunciados.
\end{obs}