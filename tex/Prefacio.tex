\section*{Prefacio}
Estas notas son una transcripción (reorganizada y con muchos añadidos) de las clases de la asignatura \ti{Geometría Lineal}, impartidas por Antonio Valdés, y sus sustitutos, entre los que se contaban un esquizofrénico que no sabía por qué oía voces y un clon de Felipe González con pendiente, en el curso 2016--2017 a los cursos de tercero de los dobles grados de matemáticas e informática y matemáticas y física en la facultad de Ciencias Matemáticas de la Universidad Complutense de Madrid (UCM).

Se han incluido demostraciones que usualmente se dan por evidentes y algunas aclaraciones de otros textos que consideramos importantes para un correcto seguimiento de una asignatura como esta.

Consideramos un requisito indispensable para seguir estas notas haber entendido bien el álgebra lineal, no obstante, se incluye un anexo con los conocimientos que consideramos indispensables, para evitar que el lector tenga que desempolvar con demasiada frecuencia la bibliografía del primer curso.
\subsection*{Agradecimientos}
Agradecemos las grandes aportaciones de Iván Prada a la hora de ilustrar este texto, así como para ayudar a limpiarlo de errores y erratas.
