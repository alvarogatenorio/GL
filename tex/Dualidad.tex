\chapter{Dualidad}
\label{C2}
El objetivo de capítulo es enunciar y justificar el llamado \ti{principio de dualidad}, tanto para espacios vectoriales como para espacios proyectivos.

No debe alarmarse el lector al que no le quedaron claros los conceptos referentes al espacio dual en un primer curso de álgebra lineal, ya que la mayoría se vuelve a explicar aquí con todo detalle.

Además, hemos evitado uno de los conceptos que suele quedar menos claro, las ``bases duales''. No obstante, este es necesario para la demostración de algunos resultados complementarios incluidos en el apéndice \ref{A}, donde se revisitan las bases duales.
\section{Dualidad en espacios vectoriales}
\label{C2_dualidadVectoriales}
Sea $E$ un espacio vectorial de dimensión finita. Llamábamos \ti{espacio dual} asociado a $E$ al conjunto de todas las aplicaciones lineales (homomorfismos de espacios vectoriales) que nacen en $E$ y mueren en $\K$. A estas aplicaciones lineales las denominamos \ti{formas lineales} o $1$--\ti{formas}.

Al espacio dual asociado a $E$ lo denotábamos por $E^*$. Expresado con una notación conjuntista:
\begin{equation}
	\label{C2_eq_defDual}
	E^*:\equals{def.}\{\alpha:E\rightarrow \K\tq \alpha \text{ es lineal }\}=\mathrm{Hom}(E,\K)
\end{equation}
Recordamos sin detenernos que la palabra \ti{espacio} no le queda grande al conjunto $E^*$ ya que es un espacio vectorial, siendo su dimensión la dimensión de $E$.
\subsection{Formas Lineales e Hiperplanos}
En esta sección iniciaremos la construcción del puente entre un espacio vectorial y su dual, tratando de ``unir'' las variedades más notables de un espacio vectorial, los hiperplanos, con otras variedades de su dual.
 
Antes de comenzar, fijemos una base $\mc{B}$ de $E$. Asimismo fijamos la base $\{1\}$ de $\K$.

Es claro que una forma lineal $h\in E^*$, tiene por matriz asociada cierta matriz $1\times n$ (respecto de las base $\mc{B}$ y $\{1\}$), a esta matriz la  denotaremos simplemente $M$.

Como ya sabemos, para cada vector $u\in E$, el valor $h(u)$ viene dado por:
\begin{equation}
	\label{C2_eq_expMatricialFormaLineal}
	h(u)=MX	
\end{equation}
siendo $X$ la matriz columna compuesta por las coordenadas de $u$ en la base $\mc{B}$. Es decir, la expresión anterior \eqref{C2_eq_expMatricialFormaLineal} no es más que el producto de una matriz fila por una matriz columna, desarrollémoslo:
\begin{equation}
	\label{C2_eq_expMatricialFormaLinealExpandida}
	h(u)=\begin{pmatrix}
	h(e_1) & \cdots & h(e_n)
	\end{pmatrix}\begin{pmatrix}
	u_1\\
	\vdots\\
	u_n
	\end{pmatrix}=h(e_1)u_1+\dots+h(e_n)u_n
\end{equation}
Como sabemos (es una comprobación inmediata), el kernel, o núcleo, de una aplicación lineal cualquiera, es un subespacio vectorial del espacio de donde nace.

Refrescamos asimismo la fórmula de las dimensiones para aplicaciones lineales, que nos será de utilidad en el futuro inmediato:
\begin{equation}
	\label{C2_eq_grassmannDimensiones}
	\dim(\ker(h))+\dim(\mathrm{im}(h))=\dim(E)
\end{equation}
Nótese que en el contexto de las formas lineales, gracias a la fórmula de las dimensiones \eqref{C2_eq_grassmannDimensiones}, los posibles valores de $\dim(\mathrm{im}(h))$ (a veces denominado \ti{rango} de $h$) son $0$ y $1$. 

En el primer caso, estaríamos ante la aplicación lineal idénticamente nula. En caso contrario, la dimensión del kernel de $h$ (a veces denominada \ti{nulidad} de $h$) sería $n-1$. Es decir, el núcleo de la forma lineal $h$ es un hiperplano de $E$.

Esto significa que $\ker(h)$ podrá ser representado mediante una única \ti{ecuación cartesiana}, ya que $\mathrm{codim}(\ker(h))=1$.

La ecuación cartesiana del núcleo de una forma lineal salta a la vista, ya que se extrae de su propia definición. No es otra que
\begin{equation}
	\label{C2_eq_ecuacionCartesianaKer}
	h(e_1)u_1+\dots+h(e_n)u_n=0
\end{equation}

Esto es evidente ya que esa es la definición del núcleo de $h$. El conjunto de aquellos vectores que, pasados por $h$, se anulan.

Acabamos de probar que el núcleo de toda forma lineal no nula es un hiperplano de $E$. Es decir, hemos construido un puente entre un espacio vectorial y su dual, identificando cada elemento del dual con un hiperplano de espacio original.

Sin embargo es un puente un tanto quebradizo, ya que solo podemos cruzarlo en una dirección (del dual al original) y, además, es posible que varios elementos del dual queden asociados al mismo hiperplano.

Para afianzar mejor este puente, veamos en primer lugar que podemos cruzarlo en el otro sentido, es decir, que todo hiperplano está asociado ``vía núcleos'' a un elemento del dual.

En términos de aplicaciones:
\begin{lem}[Pseudolema de la Correspondencia]
	\label{C2_lem_pseudocorrespondencia}
	La aplicación:
	\[\begin{array}{c}
	E^*\to\mc{H}\\
	h\mapsto \ker(h)
	\end{array}\]
	es sobreyectiva.
	
	$\mc{H}$ denota el conjunto de los hiperplanos de $E$.
\end{lem}
\begin{proof}
	Dado un hiperplano $H$, obtenemos una ecuación cartesiana que lo describa:
	\[\alpha_1x_1+\dots+\alpha_nx_n=0\]
	Donde los $\alpha_i$ representan escalares y los $x_i$ componentes de un vector respecto de la base $\mc{B}$.
		
	Sabiendo por \eqref{C2_eq_ecuacionCartesianaKer} que la ecuación cartesiana del núcleo de una forma lineal es de la forma:
	\[h(e_1)x_1+\dots+h(e_n)x_n=0\]
	Podemos construir la forma lineal asociada al hiperplano $H$ de forma explícita. Basta definir $h$ como la forma lineal que manda el $i$--ésimo vector de la base $\mc{B}$ al escalar $\alpha_i$. Es decir, la forma lineal con matriz asociada:
	\[\begin{pmatrix}
	\alpha_1 & \cdots & \alpha_n
	\end{pmatrix}\]
\end{proof}

Con el lema \ref{C2_lem_pseudocorrespondencia} hemos avanzado un paso más en la construcción del puente entre un espacio y su dual, pudiendo ir del uno al otro y del otro al uno. Sin embargo, todavía queda un problema pendiente. El elemento del dual asociado a un hiperplano no es único (ni mucho menos).

Dado un hiperplano $H$ existen infinitud de formas lineales cuyo núcleo es $H$. Esto es debido a que toda ecuación lineal equivalente a la ecuación cartesiana del hiperplano es también una ecuación cartesiana del hiperplano, a partir de la cual se construye (lema \ref{C2_lem_pseudocorrespondencia}) una forma lineal asociada a $H$ distinta de la primera.

Por ende, ahora nos interesa encontrar relaciones entre las formas lineales con idéntico núcleo, para así agruparlas y conseguir una biyección.

\begin{lem}[Lema de la Correspondencia]
	\label{C2_lem_correspondencia}
	Todas las formas lineales asociadas a un mismo hiperplano $H$ son múltiplos entre sí. En términos de aplicaciones:
	\[\begin{array}{c}
	\proy(E^*)\to\mc{H}\\
	\class{h}\mapsto \ker(h)
	\end{array}
	\]
	es biyectiva. Dicho de otra forma, los hiperplanos de un espacio vectorial se indentifican con las rectas de su dual.
\end{lem}
\begin{proof}
	Sean dos ecuaciones cartesianas de $H$:
	\begin{gather}
		H\equiv \alpha_1x_1+\dots+\alpha_nx_n=0\\
		H\equiv \beta_1x_1+\dots+\beta_nx_n=0
	\end{gather}
	Estas ecuaciones serán equivalentes, es decir, tendrán el mismo conjunto de soluciones (el hiperplano $H$). Por ende (proposición \ref{A1_prop_criterioEquivalencia}) serán múltiplos entre sí, es decir:
	\begin{equation}
		\beta_i = \lambda\alpha_i\ \forall i\in\{1,\dots,n\}
	\end{equation}
	Aplicando el método de construcción (lema \ref{C2_lem_pseudocorrespondencia}) de formas lineales asociadas a $H$ respecto de ambas ecuaciones cartesianas se obtienen:
	\begin{gather}
		f\equiv\begin{pmatrix}
			\alpha_1 & \cdots & \alpha_n
		\end{pmatrix}\\
		g\equiv\lambda\begin{pmatrix}
			\alpha_1 & \cdots & \alpha_n
		\end{pmatrix}
	\end{gather}
	Es decir $f=\lambda g$, o lo que es lo mismo $g\in\class{f}$. Esto prueba la inyectividad que queríamos, ya que la sobreyectividad se probó en el lema \ref{C2_lem_pseudocorrespondencia}.
\end{proof}

Este último resultado pone fin a la construcción del puente entre un espacio y su dual. Recapitulando, un hiperplano queda asociado, ``vía núcleos'' a una única recta su dual. Asimismo, cada recta del dual tiene asociada un hiperplano del espacio ``primal''.
\subsection{Dualidad Canónica}

En esta sección tratamos de generalizar lo dicho en el caso anterior. Es decir, trataremos de identificar variedades lineales arbitrarias con variedades lineales del dual correspondiente.

Antes de comenzar consideramos fijadas las bases $\mc{B}$ de $E$ y $\{1\}$ de $\K$, como en la sección anterior. 

En el caso de los hiperplanos, los identificábamos con el conjunto de las formas lineales cuyo núcleo era dicho hiperplano. Dicho de otra forma, el conjunto de las aplicaciones lineales que anulaban todos los vectores del hiperplano.

Siguiendo esta idea, probaremos a identificar una variedad lineal arbitraria con el conjunto de aplicaciones que anulan dicha variedad. Este conjunto es, si recordamos, nuestro viejo amigo el \ti{anulador} de la variedad.

\begin{defi}[Anulador de un subespacio vectorial]
	\label{C2_def_anulador}
	Sea $W\subset E$ subespacio vectorial de $E$. Se define el \ti{anulador} de $W$, al que denotaremos por $W^\perp$, como el conjunto formado por las $1$--formas que anulan todos los vectores de $W$. Es decir:
	\begin{equation}
	W^{\perp}:=\{\alpha\in E^*\tq \alpha(u)=0 \ \forall u\in W\}=\{\alpha\in E^*\tq W\subset ker(\alpha)\}
	\end{equation}
\end{defi}
Inmeditamente se comprueba que, el anulador de un subespacio vectorial, es un subespacio vectorial del dual asociado. 

\begin{obs}[Generalización de la Noción de Anulador]
	Se puede, generalizar la definición de anulador, no solo para subespacios sino tambien para subconjuntos arbitrarios de un espacio vectorial. La conveniencia de esta generalización viene dada porque desbloquea algunos resultados técnicos de gran utilidad. Como la importancia de esto aquí es muy reducida, tanto la generalización de la definición \ref{C2_def_anulador} como los resultados más elementales se han trasladado al apéndice. En concreto a la definición \ref{A1_def_anulador} y al lema \ref{A1_lem_anuladorBase}.
\end{obs}

Intentemos reeditar el lema de la correspondencia \ref{C2_lem_correspondencia} del apartado anterior, tratando de identificar a cada subespacio de $E$ con su anulador correspondiente.
\begin{lem}[Lema de la Correspondencia]
	\label{C2_lem_correspondenciaAnulador}
	Los subespacios de $E$ están en biyección con los subespacios de $E^*$ de la siguiente manera:
	\[\begin{array}{cc}
	\phi:& \mc{U}\rightarrow\mc{U}^*\\
	&U\mapsto U^{\perp}
	\end{array}\]
	donde $\mc{U}$ y $\mc{U}^*$ denotan el conjunto de los subespacios de $E$ y $E^*$ respectivamente.
	Dicho de otra forma, cada subespacio puede identificarse con su anulador.
\end{lem}
\begin{proof}
	Para la sobreyectividad veamos que todo subespacio $W$ de dimensión $r$ de $E^*$ es el anulador de un cierto subespacio $U$.
	
	Para esto, trataremos de probar que el conjunto de vectores de $E$ que son anulados por todas las formas lineales de $W$ (imagen inversa de $\Phi$) es un subespacio vectorial de $E$. A este conjunto le denominaremos \ti{antianulador} o \ti{anulador dual}. Este nombre se debe a que el anulador de dicho conjunto (por definición) es $W$.
	
	Notamos que $W$ admitirá una cierta base compuesta de $r$ formas lineales. Esto tiene importancia ya que cada vector que sea anulado por todas las formas lineales de la base, también lo será, por linealidad, por todas las aplicaciones de $W$.
	
	Dicho lo cual, tenemos que:
	\[W=\lengen{f_1,\dots,f_r}\]
	Dichas formas lineales tendrán ciertas matrices asociadas:
	\[f_i\equiv\begin{pmatrix}
	a_1^i & \dots & a_n^i
	\end{pmatrix}\]
	Dado un vector $x=(x_1,\dots,x_n)_{\mc{B}}$, su valor por $f_i$ viene dado por la ecuación  \eqref{C2_eq_expMatricialFormaLinealExpandida}:
	\[f_i(x)=a_1^ix_1+\dots+a_n^ix_n\]
	
	Así, pues el conjunto de los vectores tales que son anulados por las formas lineales de $W$ (antianulador) \tb{es} el conjunto de vectores que cumplen las ecuaciones:
	\begin{equation}
	\label{C2_eq_ecuacionesAtianulador}
	\left\{\begin{array}{ccc}
	a_1^1x_1+\dots+a_n^1x_n&=&0\\
	\ddots& \vdots & \vdots\\
	a_1^rx_1+\dots+a_n^rx_n&=&0
	\end{array}\right.\end{equation}
	Estas ecuaciones pueden interpretarse por las ecuaciones cartesianas de cierto subesacio $U$ de dimensión $n-r$ cuyo anulador es precisamente $W$.
	
	Esto también prueba la inyectividad, ya que hemos visto que la imagen inversa de $\Phi$ es un único subespacio.
\end{proof}
\begin{obs}[Dimensiones de los Dualizados]
	\label{C2_obs_dim_anulador}
	Obsérvese que hemos probado algo bastante importante (además de lo queríamos probar en un principio) y es que las dimensiones de un subespacio y su anulador suman la dimensión del espacio total, es decir:
	\begin{equation}\label{C2_eq_dim_anulador}
	\dim(U)+\dim(U^{\perp})=\dim(E)
	\end{equation}
	
	Dicho de otra forma, un subespacio $U$ de dimensión $r$ ``dualiza'' en un subespacio $U^\perp$ de dimensión $n-r$ (su codimensión).
\end{obs}

Recapitulando, hemos conseguido construir un puente ``sólido'' y ``de ida y vuelta'' entre las variedades de un espacio vectorial y las de su dual ``vía anuladores'' y ``vía antianuladores'' respectivamente.

Para el lector que haya optado por omitir la demostración del lema \ref{C2_lem_correspondenciaAnulador} el término ``antianulador'' resultará extraño. Definámoslo aparte:
\begin{defi}[Antianulador]
	Sea un subespacio $W$ del dual $E^*$, se define su \ti{antianulador} como el conjunto de los vectores que son anulados por todos los elementos de $W$. Expresado de forma conjuntista:
	\[W^\top:=\{u\in E\tq \alpha(u)=0\ \forall\alpha\in W\}\]
\end{defi}

Es claro que esto no es más que la imagen inversa de la aplicación del lema \ref{C2_lem_correspondenciaAnulador}, que asigna a cada subespacio de $E$ su anulador correspondiente. Como ese mismo lema demuestra que tal aplicación es biyectiva es trivial deducir que:
\begin{equation}
	\label{C2_eq_involutividad}
	(W^{\perp})^{\top}=W
\end{equation}

\begin{obs}[Abusos de Notación Habituales]
	A veces, el antianulador se denota como el anulador. Además, para dar verosimilitud a este abuso de notación, se define el anulador de un subespacio de $E^*$ como lo que nosotros conocemos como antianulador. Dicho abuso de notación permite hacer afirmaciones como:
	\[(W^\perp)^\perp=W\]
	Lo cual, a priori, con nuestra notación, no es cierto.
	
	Sin embargo, trabajando con cuidado, podemos demostrar que, efectivamente, con nuestras notaciones se cumple que el operador anulador es ``esencialmente involutivo''. Es decir:
	\begin{equation}
		(W^\perp)^\perp\cong W
	\end{equation}
	Esto significa que existe un isomorfismo entre el anulador del anulador un subespacio (un subespacio del dual del dual) y el subespacio en sí. No solo eso, se puede demostrar que dicho isomorfismo es especialmente agradable, por lo que se le da el nombre de \ti{canónico}.
	
	La prueba de este hecho se ha trasladado al apéndice \ref{A}.
\end{obs}
Veamos a continuación una serie de propiedades que nos serán de gran ayuda cuando conozcamos el llamado ``principio de dualidad'', en la sección \ref{C2_principioDualidadLineal}. Estas propiedades pueden resumirse en dos; la inversión de las contenciones y las leyes de DeMorgan.
\begin{prop}[Propiedades del Anulador]
	\label{C2_pro_propiedades_dualidad}
	Se cumplen las siguientes propiedades:
	\begin{enumerate}
		\item Los contenidos se invierten al dualizar. Es decir: \[W\subset U\sii U^{\perp}\subset W^{\perp}\]
		\item Las sumas se convierten en intersecciones al dualizar:
		\[(U+W)^{\perp}=U^{\perp}\cap W^{\perp}\]
		\item Las intersecciones se convierten en sumas:
		\[(U\cap W)^{\perp}=U^{\perp}+ W^{\perp}\]
	\end{enumerate}
\end{prop}
\begin{proof}
	\begin{enumerate}
		\item \begin{itemize}
			\item[$\bra$] Dado un $\alpha\in U^\perp$, veamos que $\alpha\in W^\perp$. En efecto, $\alpha(u)=0$ para todo $u\in U$, pero como $W\subset U$ se tiene que $\alpha(w)=0$ para todo $w\in W$, luego $\alpha\in W^\perp$.
			
			\item[$\bla$] Sea $w\in W$, veamos que $w\in U$. Es claro que $\alpha(w)=0$ para toda $\alpha \in W^{\perp}$. Como $U^{\perp}\subset W^{\perp}$ se tiene que $\beta (w)=0$ para toda $\beta\in U^{\perp}$. Luego $w$ pertenece al antianulador del anulador de $U$. Por la ecuación \eqref{C2_eq_involutividad} se tiene que $w\in (U^\perp)^\top=U$.
		\end{itemize}
		
		
		\item Sea $\alpha\in(U+W)^\perp$, inmediatamente se desprende que $\alpha(u+w)=\alpha(u)+\alpha(w)=0$ para todo $u\in U$ y todo $w\in W$. Como $u\in U+W$ y $w\in U+W$ tenemos que $\alpha(u)=0$ para todo $u\in U$ y $\alpha(w)=0$ para todo $w\in W$. Luego $\alpha$ pertenece a los anuladores de $U$ y $W$ simultáneamente. Es decir, $\alpha\in U^\perp\cap W^\perp$. Como todos los pasos que hemos hecho son equivalencias, el resultado se sigue.
		\item Dado $\alpha\in U^\perp + W^\perp$, veamos que $\alpha\in (U\cap W)^\perp$. Usando que $\alpha\in U^\perp + W^\perp$ tenemos que $\alpha=\beta+\gamma$ donde $\beta\in U^\perp$ y $\gamma\in W^\perp$. Para probar que pertenece al anulador de la intersección, tomemos un $\xi\in U\cap W$ arbitrario y veamos que $\alpha$ lo anula. En efecto:
		\[\alpha(\xi)=(\beta+\gamma)(\xi)=\beta(\xi)+\gamma(\xi)=0\]
		Acabamos de ver que $U^\perp+W^\perp\subset(U\cap W)^\perp$. Para ver la igualdad, veamos que ambos tienen la misma dimensión. Para ello usaremos la fórmula de Grassmann, el apartado anterior de esta demostración y la observación \ref{C2_obs_dim_anulador}:
		\begin{multline}
			\dim(U^\perp+W^\perp)=\dim(U^\perp)+\dim(W^\perp)-\dim(U^\perp\cap W^\perp)=\\
			=\dim(U^\perp)+\dim(W^\perp)-\dim((U+ W)^\perp)=\\
			=n-(\dim(U)+\dim(W)-\dim(U+W))=\\
			=n-(\dim(U\cap W))=\\
			=\dim((U\cap W)^\perp)
		\end{multline}
	\end{enumerate}
	Con lo que concluye la demostración.
\end{proof}

A la identificación de un subespacio con su anulador se la denomina \ti{dualidad canónica}.
\subsection{Principio de Dualidad}
\label{C2_principioDualidadLineal}

Estamos en disposición de enunciar el llamado \ti{principio de dualidad}, un resultado de extrema importancia ya que, por cada teorema que demostremos obtendremos otro sin necesidad de demostrarlo. A continuación trataremos de enunciar y justificar este principio de forma natural, con un ejemplo. Presentamos los siguientes enunciados:
\begin{theo}
	\label{C2_teo_principioDualidad1}
	Sea un espacio vectorial de dimensión $3$ entonces:
	
	Dos rectas distintas generan un plano.
\end{theo}
\begin{theo}
	\label{C2_teo_principioDualidad2}
	Sea un espacio vectorial de dimensión $3$ entonces:
	
	Dos planos distintos se cortan en una recta.
\end{theo}
A simple vista los teoremas \ref{C2_teo_principioDualidad1} y \ref{C2_teo_principioDualidad2} parecen dos teoremas totalmente independientes de tal forma que cada cual requerirá una prueba.

Bien, tratemos de demostrar el teorema \ref{C2_teo_principioDualidad1}. (Recomendamos encarecidamente al lector no omitir la siguiente demostración).
\begin{proof}[Demostración del Teorema \ref{C2_teo_principioDualidad1}]
	Nos preguntamos si la suma de dos subespacios de dimensión uno cuya intersección es un espacio de dimensión nula tendrá dimensión $2$. Es decir:
	\begin{equation*}
		\dim(U+W)\stackrel{\mathrm{?}}{=}2
	\end{equation*}
	
	Como por la observación \ref{C2_obs_dim_anulador} preguntarse por la dimensión de un subespacio es preguntarse por la dimensión de su anulador obtenemos que nuestra pregunta inicial es equivalente a:
	\begin{equation*}
		\dim((U+W)^\perp)\stackrel{\mathrm{?}}{=}3-2=1
	\end{equation*}
	
	Asimismo, por la proposicion \ref{C2_pro_propiedades_dualidad} sabemos que el anulador de una suma es la intersección de anuladores, luego esta segunda pregunta es equivalente a:
	\begin{equation*}
		\dim((U^\perp\cap W^\perp))\stackrel{\mathrm{?}}{=}1
	\end{equation*}
	donde, por la observación \ref{C2_obs_dim_anulador} tenemos que $\dim(U^\perp)=\dim(W^\perp)=2$.
	
	Además, como $\dim(U\cap W)=0$, se tiene que $\dim(U^\perp+W^\perp)=3$.
	
	Así, a partir de nuestra primera pregunta, hemos obtenido una equivalente que reza:
	
	\ti{¿En un espacio de dimensión $3$, dos planos distintos se cortan en una recta?}
	
	Y esta es, precisamente la pregunta que deberíamos hacernos si estuviéramos tratando de demostrar el teorema \ref{C2_teo_principioDualidad2}.
	
	Con esto hemos demostrado que los teoremas \ref{C2_teo_principioDualidad1} y \ref{C2_teo_principioDualidad2} son equivalentes. Es decir, si uno es cierto, es cierto el otro (y viceversa). Asimismo, si uno es falso, el otro también lo será (y al revés). Por ende, con probar uno de los dos nos valdrá.
	
	En lo que respecta a la prueba del teorema, es un cálculo inmediato con la fórmula de Grassmann y se deja como ejercicio de cálculo mental al lector.
\end{proof}

Quizá este no sea el mejor ejemplo para apreciar la gran utilidad de este principio, ya que, las demostraciones de ambos teoremas son extraordinariamente sencillas. Sin embargo, pueden darse casos (y se darán a lo largo del texto) en los que la demostración de un teorema sea extraordinariamente sencilla en el caso dual y algo más engorrosa en el caso ``normal''.

Así pues, dado cierto aserto sobre espacios vectoriales compuesto en términos de sumas, intersecciones y contenidos puede traducirse a un \ti{aserto dual} equivalente gracias a las propiedades demostradas en la sección anterior.

Esto es una auténtica fábrica de teoremas, ya que, si demostramos la veracidad de un enunciado, obtendremos automáticamente la veracidad de su aserto dual equivalente. 
\section{Dualidad en espacios proyectivos}
Una vez repasados y ampliados los conceptos de dualidad en espacios lineales, pasemos a introducir la dualidad en espacios proyectivos. Como siempre, iremos trasladando al contexto proyectivo los resultados del mundo lineal.
\begin{defi}[Espacio Proyectivo Dual]
	Dado un espacio vectorial $E$, se llama \ti{espacio proyectivo dual} de $E$ al espacio proyectivo asociado al espacio vectorial dual $E^*$.
	
	Lo denotaremos, de forma natural, por $\proy(E^*)$.
\end{defi}
\begin{obs}[Dimensión del Espacio Proyectivo Dual]
	Dado que (si la dimensión es finita) el espacio dual $E^*$ tiene la misma dimensión que $E$, deducimos inmediatamente que la dimensión del espacio proyectivo es la misma que la del espacio proyectivo dual. Es decir:
	\begin{equation}
	\dim(\proy(E))=\dim (\proy(E^*))
	\end{equation}
\end{obs}

\subsection{Formas Lineales e Hiperplanos Proyectivos}

Comenzamos este apartado recordando brevemente que un hiperplano vectorial no es otra cosa que un subespacio lineal de codimensión $1$. Asimismo, un hiperplano proyectivo $H$ no es más que el espacio proyectivo asociado a cierto hiperplano vectorial $\widehat{H}$. Es trivial observar que la dimensión de un hiperplano proyectivo es $\dim(\proy(E))-1$. Luego son variedades de codimensión $1$ respecto de la dimensión del espacion proyectivo total.\\

Recordemos un importante resultado obtenido en el lema \ref{C2_lem_correspondencia}, que afirma que todo hiperplano  vectorial $\widehat{H}$ está en biyección con las formas lineales cuyo núcleo es el propio $\widehat{H}$, las cuales conforman un rayo de $E^*$. Es decir, todo hiperplano está en biyección con un punto del espacio proyectivo dual.

Tratamos ahora de fabricarnos un puente entre los hiperplanos proyectivos y alguna variedad del espacio proyectivo dual, a imagen y semejanza de lo hecho en secciones anteriores.

Surge así el siguiente lema, casi idéntico al lema~\ref{C2_lem_correspondencia}.
\begin{lem}[Lema de la Correspondencia Proyectiva]
	\label{C2_lem_correspondenciaProy}
	La aplicación $\psi=\pi\circ\varphi$, donde $\varphi=\Phi\circ \pi^{-1}$, es biyectiva.
	\begin{equation*}
		\xymatrix{
			\mc{H} \ar[r]^{\pi^{-1}} \ar[rd]_{\varphi\equiv\Phi\circ\pi^{-1}} & \widehat{\mc{H}}\ar[d]^{\Phi\equiv\widehat{H}^{\perp}}\\
			 & E^* \ar[r]_{\pi} & \proy(E^*)
		}
	\end{equation*}
	Aclarando la notación, $\mc{H}$ denota el conjunto de los hiperplanos proyectivos de $\proy(E)$, $\pi^{-1}$ representa la imagen inversa de la proyección canónica y $\Phi$ es la dualidad canónica.
\end{lem}
\begin{proof}
	El conjunto de los hiperplanos proyectivos de $\proy(E)$ está en biyección vía $\pi^{-1}$ con el conjunto de los hiperplanos vectoriales de $E$ (observación \ref{C1_obs_lemaCorrespondencia}).
	
	A su vez, por el lema \ref{C2_lem_correspondencia}, los hiperplanos vectoriales están en biyección con las rectas del espacio dual vía dualidad canónica.
	
	Asimismo, las rectas del dual están en biyección con los puntos del espacio proyectivo dual vía proyección canónica $\pi$.
	
	Dado que la composición de biyecciones es biyección, queda demostrado que los hiperplanos proyectivos están en biyección con los puntos del espacio proyectivo dual.
\end{proof}
Entonces, dado un hiperplano proyectivo $H$, podemos escribirlo de la forma:
\begin{equation}
H=\proy(\hat{H})=\{[u]\in\proy(E)\tq h(u)=0\}
\end{equation}

\subsection{Dualidad Canónica}
Hemos conseguido identificar los hiperplanos proyectivos con puntos del espacio proyectivo dual. Al igual que hicimos en la sección anterior, tratemos ahora de generalizar ese resultado para variedades proyectivas arbitrarias.

Recordemos que, dada una variedad lineal, esta estaba en biyección con su anulador, el cual es una variedad lineal de $E^*$. Trasladando de forma natural esta idea al contexto proyectivo obtenemos el siguiente resultado.
\begin{lem}[Lema de la Correspondencia Proyectiva]
	\label{C2_lem_correspondencia_proy_anulador}
	La aplicación $\psi=\pi\circ\varphi$ es biyectiva.
	\begin{equation*}
		\xymatrix{
			\mc{V} \ar[r]^{\pi^{-1}} \ar[dr]_{\varphi\equiv\Phi\circ\pi^{-1}} & \widehat{\mc{V}} \ar[d]^{\widehat{V}^\perp\equiv\Phi} & \\
			& \widehat{\mc{V}^*} \ar[r]_{\pi} & \mc{V}^* 
		}
	\end{equation*}
	Aclarando la notación, los conjuntos $\mc{V}$, $\widehat{\mc{V}}$, $\widehat{\mc{V}^*}$ y $\mc{V}^*$ denotan el conjunto de las variedades de $\proy(E)$, $E$, $E^*$ y $\proy(E^*)$ respectivamente.
	
	Así, dada una variedad proyectiva $\mc{X}$, se corresponde con su \ti{dualizada} $X^*=\psi(X)=\proy(\widehat{X}^\perp)$
\end{lem}
\begin{proof}
	Es idéntica a la del lema \ref{C2_lem_correspondenciaProy}.
\end{proof}

Obsérvese que, lo que estamos haciendo, no es otra cosa que identificar cada variedad proyectiva $X=\proy(\hat{X})$ con el proyectivizado del anulador de $\widehat{X}$.

\begin{obs}[Caso Particular de los Hiperplanos]
	Veamos, a modo de aclaración suplementaria, que el lema \ref{C2_lem_correspondencia_proy_anulador} funciona también (por supuesto), para hiperplanos.
	\begin{equation*}
	H^*=\proy(\hat{H}^{\perp})=\proy(\{h\in E^*\tq h(u)=0\ \forall u\in \hat{H}\})=[h]
	\end{equation*}
\end{obs}
Las propiedades que se desprendían del lema \ref{C2_lem_correspondencia} pueden extenderse sin esfuerzo al contexto proyectivo, tal y como muestra la siguiente proposición.
\begin{prop}[Propiedades de la Dualidad Proyectiva]
	\label{C2_pro_propiedades_dualidad_proy}
	Sean $E$ un espacio vectorial y su correspondiente espacio proyectivo $\proy(E)$. Sean $X,Y\subset\proy(E)$ variedades proyectivas. Se cumple:
	\begin{enumerate}
		\item $X\subset Y\sii Y^*\subset X^*$
		
		\item $(X\cap Y)^*=\engen{X^*,Y^*}$
		
		\item $\engen{X,Y}^*=X^*\cap Y^*$
		
		\item $\dim(X)+\dim(X^*)=\dim(\proy(E))-1$
	\end{enumerate}
\end{prop}
\begin{proof}
	\begin{enumerate}
		\item Si $X\subset Y$, entonces $\hat{X}\subset\hat{Y}$. Por la proposición~\ref{C2_pro_propiedades_dualidad} tenemos que $\hat{Y}^{\perp}\subset \hat{X}^{\perp}$, y, por tanto, $Y^*\subset X^*$. Recíprocamente, basta leer la demostración al revés.
		
		\item Por el lema~\ref{C1_lem_interseccionVariedades} se tiene que $X\cap Y=\proy(\hat{X}\cap\hat{Y})$. Por tanto, $(X\cap Y)^*=\proy((\hat{X}\cap\hat{Y})^{\perp})$. Aplicando la proposición~\ref{C2_pro_propiedades_dualidad} se tiene que $\proy((\hat{X}\cap\hat{Y})^{\perp})=\proy(\hat{X}^{\perp}+\hat{Y}^{\perp})=\engen{X^*,Y^*}$.
		
		\item Se tiene que $\engen{X,Y}=\hat{X}+\hat{Y}$. Por tanto, $\engen{X,Y}^*=\proy((\hat{X}+\hat{Y})^{\perp})$. Por la proposición~\ref{C2_pro_propiedades_dualidad} sabemos que $\proy((\hat{X}+\hat{Y})^{\perp})=\proy(\hat{X}^{\perp}\cap\hat{Y}^{\perp})$. Atendiendo de nuevo al lema~\ref{C1_lem_interseccionVariedades} queda que $\proy(\hat{X}^{\perp}\cap\hat{Y}^{\perp})=\proy(\hat{X}^{\perp})\cap\proy(\hat{Y}^{\perp})=X^*\cap Y^*$
		
		\item Se tiene inmediatamente:
		\begin{equation*}
		\dim(X)+\dim(X^*)=\dim(\hat{X})-1+\dim(E)-\dim(\widehat{X})-1=\dim(\proy(E))-1
		\end{equation*}
	\end{enumerate}
\end{proof}
\begin{obs}
	El lema anterior reafirma, aún más si cabe, que el dualizado de un hiperplano proyectivo es un punto. En efecto supongamos que $\dim(E)=m+1$, entonces:
	\begin{equation*}
	\dim(X)+\dim(X^*)=m-1
	\end{equation*}
	Como $\dim(X)=m-1$, se sigue que $\dim(X^*)=0$.
\end{obs}

\subsection{Principio de Dualidad para espacios proyectivos}
\label{C2_principioDualidadProyectiva}
Llegados a este punto parece natural tratar de extrapolar el principio de dualidad presentado en la sección \ref{C2_principioDualidadLineal} al mundo proyectivo, ya que contamos con todos los ingredientes necesarios.

La proposición \ref{C2_pro_propiedades_dualidad_proy} es de vital importancia ya que, literamente, es una ``fábrica de enunciados equivalentes''. Expliquemos esto.\\

Supongamos que nos preguntamos por la dimensión de cierta variedad proyectiva de un espacio proyectivo de dimensión $n$. La proposición \ref{C2_pro_propiedades_dualidad_proy} (en su cuarto apartado) afirma que decir que cierta variedad $X$ tiene cierta dimesnión $r$ es equivalente a decir que su variedad dualizada $X^*$ tendrá dimensión $n-1-r$.

Esto quiere decir que si una afirmación es cierta, la otra también lo será (y viceversa), cumpliéndose algo similar si alguna de las dos afirmaciones es falsa. Por ende, con demostrar o refutar una de las dos afirmaciones (la que nos resulta más sencilla), habremos refutado o demostrado la otra. Esta es la esencia del principio de dualidad, tener dos teoremas al precio de uno (el más barato).\\

El otro ámbito de aplicación del principio de dualidad es cuando nos preguntamos si cierta variedad $X$ está contenida dentro de cierta otra variedad $Y$. En este caso, la proposición \ref{C2_pro_propiedades_dualidad_proy} (en su primer apartado) nos dice que afirmar que $X\subset Y$ es equivalente a afirmar que $Y^*\subset X^*$.

La combinación de estos dos ámbitos de aplicación, unidos a las prodiedades auxiliares que nos ofrece la proposición \ref{C2_pro_propiedades_dualidad_proy} (en su segundo y tercer apartados) es lo que hace  realmente útil y potente al principio de dualidad. Veamos algunos ejemplos.

\begin{exa}[Obtención del Enunciado Dual]
	Hallemos la proposición dual de ``\ti{en un plano proyectivo real toda recta contiene al menos tres puntos diferentes}''.
	
	Como vemos, nos preguntamos si cierta variedad $r$ (recta) contiene a cierta variedad $p_i$ (punto). Es decir:
	\[p_i\subset r\]
	Esto es equivalente a preguntarse si el dualizado de $r$, al que llamaremos $r^*$ está contenido en el dualizado de $p_i^*$.
	\[r^*\subset p_i^*\]
	Como $r$ es una variedad de dimensión $1$ en un espacio de dimensión $2$, $r^*$ será una variedad de dimensión $0$ (un punto). Análogamente, al ser $p_i$ un punto, $p_i^*$ será una recta.
	
	Así pues el enunciado dual (equivalente al primero) será ``\ti{Por todo punto pasan al menos $3$ rectas diferentes}''.
	
	Nótese que no hemos demostrado la veracidad ni la falsedad de ninguno de los enunciados, sin embargo, con demostrar algo sobre uno de ellos, este algo será automáticamente válido para el otro.
\end{exa}
Todos estos puentes, identificaciones, y demás no serían de ninguna utilidad si no nos permitiesen resolver problemas mayor facilidad. Hasta ahora no hemos visto ningún ejemplo que realmente haga valer todo nuestro trabajo. Pues, al fin y al cabo, ¿no podemos simplemente resolver los problemas ``a pelo''? Es posible, sí, pero muchas veces hacer la asociación entre una variedad proyectiva y su dual, es decir la proyección del anulador, facilita enormemente la resolución. Veamos a continuación el ejemplo prometido.
\begin{exa}[Resolución del Problema Dual]
	Sean dos rectas (variedades de dimensión $1$) $r_1,r_2$ en el espacio proyectivo $\proy^3=\proy(\R^4)$, las cuales no se cortan. Asimismo sea un punto $p\in\proy^3$ no perteneciente a ninguna de las rectas. Demuestre que existe una única recta $r$ que pasa por $p$ y corta a ambas rectas $r_1, r_2$.
	
	Desbrozando la literatura del enunciado tenemos dos rectas $r_1,r_2$ y un punto $p\in\proy^3$ tales que $r_1\cap r_2=\emptyset$ y $p\not\in r_1\cup r_2$. Debemos probar que existe una única recta $r$ tal que $p\in r$, $r_1\cap r\not=\emptyset$ y $r_2\cap r\not=\emptyset$. Resolvamos el problema primero sin dualizar, y luego pasando al dual.
	\begin{enumerate}
		\item Tomemos la variedad proyectiva engendrada por $r_1$ y $p$, la cual es un plano ya que
		\begin{equation*}
		\dim(\engen{p,r_1})=\dim(p)+\dim(r_1)-\dim(r_1\cap p)=0+1-(-1)=2
		\end{equation*}
		Podemos aplicar el corolario~\ref{C1_cor_rectaHiperplano} al plano $\engen{p,r_1}$ y la recta $r_2$, según el cual una recta y un hiperplano siempre se cortan. Antes, y para obtener el resultado deseado, debemos asegurarnos de que $r_2\not\subset\engen{p,r_1}$, pues en caso contrario existirían más de un punto de corte entre la recta y el hiperplano y $r$ no sería única. Es fácil comprobar que esto no ocurre, ya que si $r_2\subset\engen{p,r_1}$, entonces $r_1\cap r_2\not=\emptyset$, llegando así a un absurdo. Existirá, por tanto, un único punto $q\in r_2\cap\engen{p,r_1}$. Definimos entonces la recta $r$ como la variedad engendrada por los puntos $p$ y $q$, pudiéndose comprobar con la fórmula de las dimensiones que efectivamente es una recta. Nótese que $r$ es única, ya que lo es el punto $q$. Además $r_1\cap r\not=\emptyset$ y $r_2\cap r\not=\emptyset$, ya que $q\in r_2\cap\engen{p,r_1}$. Queda así demostrado el ejercicio.
		
		\item Para empezar, y atendiendo a la proposición~\ref{C2_pro_propiedades_dualidad_proy}, la ecuación de las dimensiones que caracteriza la dualización es, en nuestro caso:
		\begin{equation*}
		\dim(X)+\dim(X^*)=2. 
		\end{equation*}
		Por tanto, el dual de un punto es un plano y el dual de una recta, otra recta. Tenemos entonces que $p^*$ es un plano y $r_1^*,r_2^*$ son rectas. Por otro lado, que $p\in r$ implica, por la proposición~\ref{C2_pro_propiedades_dualidad_proy}, que $r^*\subset p^*$. De igual forma que $p\not\in r_1\cup r_2$ implica que $r_1^*\not\subset p^*$ y $r_2^*\not\subset p^*$. Además, se comprueba inmediatamente (dualizando y aplicando la fórmula de Grassmann) que como $r_1\cap r_2=\emptyset$, entonces $r_1^*\cap r_2^*=\emptyset$.
		
		Por tanto, el enunciado del problema se traduce en, dadas dos rectas $r_1^*, r_2^*\in\proy^{3^*}$ y un plano $p^*\in\proy^{3^*}$ tales que $r_1^*\cap r_2^*=\emptyset$, $r_1^*\not\subset p^*$ y $r_2^*\not\subset p^*$; demostrar que existe una única recta $r^*$ tal que $r_1^*\cap r^*\not=\emptyset$, $r_2^*\cap r^*\not=\emptyset$ y $r^*\subset p^*$.
		
		Dado que las rectas $r_1^*, r_2^*$ no están contenidas en el plano $p^*$, cortarán con él en dos puntos únicos. Es claro que la recta engendrada por esos dos puntos es única y cumple las condiciones requeridas.
	\end{enumerate}
	En este caso, la dualización del problema lo convierte en algo trivial de resolver.
\end{exa}
\begin{obs}
	Este enunciado es falso en el contexto de la geometría afín. Podemos encontrar dos rectas paralelas, $r_1$ y $r_2$, y un punto $p$, que cumplan las hipótesis del enunciado, para los cuales no existe ninguna recta $r$; o bien para los cuales existan infinitas rectas $r$, tales que $p\in r$, $r_1\cap r\not=\emptyset$ y $r_2\cap r\not=\emptyset$. Ello se debe a que en el espacio afín dos rectas paralelas no se cortan, mientras que en el espacio proyectivo sí.
\end{obs}
